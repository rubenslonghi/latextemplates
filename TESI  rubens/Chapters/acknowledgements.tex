\chapter*{Acknowledgements}
\label{acknowledgements}
\addcontentsline{toc}{chapter}{Acknowledgements}
\thispagestyle{plain}

Ebbene s\`i, sono riuscito a scrivere correttamente la parola \emph{acknowledgements}, ma questo non implica che io abbia la voglia di scriverli in inglese. Infatti non lo far\`o.\\
Solitamente, giunti alla fine del corso di laurea, si ringraziano parenti, amici, professori. Ricordarli e ringraziarli tutti sarebbe impossibile, anche perch\`e in questi tre anni credo di averne passate talmente tante insieme a tante persone, che la mia povera testolina ha dovuto rimuoverne la maggior parte per fare spazio alle ben meno importanti formule matematiche.\\

\noindent Cerchiamo di mettere ordine. Rubens Longhi approda a Pavia nel lontano 2015, dopo un'estate di delusioni da test d'ingresso falliti. Maledetto sia Edipo. A sostenermi in questo momento difficile c'\`e la decennale amica Michela, con la quale divido un appartamento per il primo anno di universit\`a. Durante quell'anno molti sono i nuovi amici, sebbene il tornare a casa ogni finesettimana mi permette di mantenere vive le amicizie di sempre.\\
L'inizio non \`e facile. Ricordo i primi giorni a fisica: io, Pozzo e Mondini emarginati in un angolo, che tentavamo di seguire il Gila senza troppo interagire con il restante centinaio di persone, tra le quali si potevano individuare quegli 8 disagiati del Ghislieri. Mamma mia quanto se la tiravano. Soprattutto un terroncello di nome Angelo e quell'altro torrone di Magoni che, dopo aver affrontato con me i test d'ingresso, mi aveva abbandonato come si fa coi cani in tangenziale. A proposito di Cane, il mio primo ricordo di Amodio Carleo si colloca in una lezione di Algebra lineare durante la quale Carleo Amodio (ma qual'\`e il nome?) sviene platealmente a seguito di bravate della notte precedente che gli hanno lasciato escoriazioni sulle ginocchia; anche se detto cos\`i suona male. (By the way, Pirola and\`o nel pallone e chiese ad un'aula gremita di fisici e matematici se ci fosse un dottore). Inizio quindi a chiedermi cosa mai faranno in quel collegio per giungere al punto di svenire a lezione. Trovai solo molto pi\`u tardi la risposta.\\
Dopo pochi giorni entro in contatto con le persone che sarebbero stati al centro del mio primo anno qui: Ale Triple, Monte, Nicole, Bonfo, Vix, Piazza e poi Bressi. I mercoled\`i sera con loro erano diventati una tappa obbligata della settimana pavese. Benedetto fu l'ingresso gratis al Camillo.\\
L'amicizia con Ale Triacca si rivela importante: accomunati dal fatto di essere dei morti di figa, durante il primo anno facciamo strage di ragazzine al Camillo. Cio\`e, lui fa strage e io finisce che bevo e ballo in maniera ridicola (s\`i, avete presente come ballo).\\
Il marted\`i, invece, diventa il giorno del vino di dubbia qualit\`a che Berti portava nel nostro appartamento nell'evidente tentativo di far ubriacare la Michela e poi sedurla. Ma la Michela, almeno ai tempi, non beveva. Inoltre il gruppo di Amaldini fuori sede a Pavia, tra cui Roby, Nicole, Losa e Pozzo si organizza con serate a tema \emph{panterona} (gli interessati sanno di chi sto parlando) nelle quali io e Pozzo diamo del nostro peggio. \\
Gli esami del primo anno (a parte Chimica, che fa cagare) vanno abbastanza bene da indurmi a fare la follia: spinto anche dai consigli della Marveggio (con la quale solevo andare a bere una birra dopo ogni esame nell'evidente tentativo di sedurla; ma anche lei, ai tempi, non beveva), decido di rifare il test d'ingresso per il Ghislieri.\\
Stavolta l'esito \`e positivo: sia benedetto Rotondi che ricicl\`o gli esercizi dell'esame di Meccanica per la prova Iuss. In quel momento tante cose cambiano. Il legame con i miei compagni d'anno collegiali diventa molto forte, anche grazie al kulo, e mi fa quasi dimenticare di essere un anno in ritardo. La mia vita viene completamente assorbita dal collegio, tanto che i periodi che passo a casa iniziano a ridursi davvero al minimo sindacale. Delle amicizie bergamasche si salvano solo quella immortale con mio cugino Daniel e quelle con i miei compagni del liceo, specialmente Pizzo, Giordano, Giulia, Giovanni, Menni e, quando c'\`e da far serata ignorante, anche Lanceni. Le cene con Berti, non essendo pi\`u in appartamento, deficitano della presenza della Michela e quindi lui \`e costretto a tentare di sedurre me.\\
La mia vita diventa un tuttuno con quella del mio vicino di stanza Djerme, giovane padawan nell'arte dell'analisi. Il fortissimo legame che si \`e creato tra di noi mi ha permesso di superare molte delle difficolt\`a del secondo anno: specialmente il maledetto Elettro 2. Il caff\`e di Allegria diventa il covo di fisici e matematici come Martina, Zio Sam, Vitto, Mago, Cane, Nino e anche di altri futuri disoccupati come Maria Ganja, Francesca (quella di Nino),  Cazzato e Virginia.\\
L\`i scopro che quel terroncello cazzone di Angelo possiede, inspiegabilmente, un cervello e che esso lavora e impara le cose esattamente come lo fa il mio: ovvero completamente a caso, senza una particolare tecnica mnemonica e senza uno studio sistematico. Questo disordine mentale comune ci porta ad affrontare molti esami insieme e a condividere mille momenti anche sui social, fino al punto da diventare decisamente rompicoglioni (questa cosa mi mancher\`a parecchio).\\
I corsi del secondo anno sono decisamente stimolanti, anche e soprattutto perch\`e non li frequento, dato che chi \`e in collegio sa bene che nei primi mesi svegliarsi in tempo al mattino non \`e sempre facile. Per fortuna (o per sfortuna) le esercitazioni di Meccanica Razionale, tenute da tale Claudio Dappiaggi, sono al pomeriggio. Di quelle lezioni mi \`e rimasta la frase \emph{Non crediate che i vostri professori bevano meno di voi}, la quale mi ha fatto capire di essermi ben instradato nel ruolo del fisico teorico. Al secondo anno incontriamo il Dappia in ben due esami e gi\`a l\`i iniziamo a pensare che ci perseguiti.\\
Il viaggio al \emph{Cern}, per quanto stimolante, non mi fa recedere dai miei insani propositi teorici, anche se la coppia di prof sperimentaloni Negri+Rebuzzi ci fa vivere emozioni incredibili, come quando la Rebu ha detto a Nino di buttarsi dal trampolino nel lago di Ginevra, e lui ha eseguito.\\
Pian piano si giunge al terzo anno, che inizia con il prevedibile fervore di noi fagioli per l'arrivo delle nuove matricole. Principalmente il mio interesse \`e rivolto a torturare le matricole con quesiti impossibili di analisi matematica, ma la situazione cambia decisamente quando, in preda ad un attacco di socializzazione acuta conosco Francesca (non quella di Nino). Le circostanze del nostro primo vero incontro sono decisamente imbarazzanti. La sua cecit\`a la spinge alla cleptomania nei confronti dei miei occhiali, alla quale io reagisco insultando la sua terribile facolt\`a. Dopo nove mesi insieme, lei ancora mi ruba gli occhiali e io ancora la insulto perch\`e fa Chimica.\\
Reincontriamo di nuovo il Dappia, che tiene per la prima volta a noi un corso al terzo anno. Come se non bastasse, ce lo ritroviamo in un corso Iuss e persino in birreria al sabato sera. A questo punto pensiamo al complotto.\\
Procede tutto bene fino al momento in cui scopro che Nino mi ha battuto sul tempo: ha gi\`a chiesto lui la tesi al Dappia. Maledetto. Di reazione, la chiedo anche io al Dappia, che mi affida ad uno dei suoi scagnozzi Nicol\`o. Al ch\`e, Nino decide di svendersi agli sperimentali, prima rimanendo invischiato in una tesi sperimentale sull'inutile massa del bosone chicchessia e poi spostandosi sulla fenomenologia della forza debole. Perch\`e Nino \`e un debole. Fattosta che a fine maggio mi rimangono ancora cinque esami da dare (eh si, quello con la Rimoldi per me \`e valso doppio) e della tesi ho scritto a malapena il primo capitolo. Ma l'importante, a questo punto, \`e salvare la faccia. Tutti i miei compagni se ne vanno verso altri, floridi, lidi e io manco sono capace di laurearmi a luglio. All'alba del 20 giugno, con ancora un esame da 12 crediti da dare e a sole tre settimane dalla consegna della tesi, decido di fare la follia. Chiss\`a quante maledizioni mi hanno tirato il Dappia e Nicol\`o quel giorno. Ma se sono qui ora, vuol dire che ce l'ho fatta.\\
\thispagestyle{plain}

\noindent Pi\`u che ringraziamenti, ho molte scuse da fare. Mi scuso per aver trascurato in questi ultimi tempi tante persone che si meritavano molto di pi\`u da parte mia. Ho trascurato la mia famiglia, non tornando a casa quasi mai. Ho trascurato molti di quegli amici che ho citato prima. Ho trascurato la mia salute, mentale e fisica.\\ Ora \`e tutto finito, ma se pensate che mi dedicher\`o a voi fin da subito, vi sbagliate di grosso. Ora ho solo voglia di andare al mare. Ci vediamo a settembre, \emph{bitches}!
\thispagestyle{plain}










