\chapter*{Conclusions}
\addcontentsline{toc}{chapter}{Conclusions}

In this thesis we established, for a particular class of spacetimes with timelike boundary and for suitable boundary conditions, the existence of advanced and retarded fundamental solutions or Green operators for Maxwell's equations stated both in terms of the Faraday tensor $F\in\Omega^2(M)$ and in terms of the vector potential $A$. Subsequently we constructed the optimal quantum algebra of observables for the free electromagnetic field in terms of $A$ for two selected boundary conditions.

In particular, for the equations in terms of $F\in\Omega^2(M)$, we analyzed Maxwell's equations in a framework in which the spacetime $M$ could be split into $\mathbb{R}\times\Sigma$, with $\Sigma$ being a closed Riemannian manifold with a codimension $1$ interface $Z\subset\Sigma$. We separated the equations in a non-dynamical and in a dynamical part. The former has been treated using the so-called Hodge decomposition, while for the latter the fundamental solutions have been constructed using the technique of {Lagrangian subspaces}. These are at the hearth of a method to select boundary conditions that ensure the self-adjointness of the dynamical part of Maxwell's equations. Furthermore we gave an example of such conditions.

In the case of Maxwell's equations for a generic vector potential $A\in\Omega^k(M)$, $0<k<\dim M$ we proceeded as follows. At first we proved of the existence of advanced and retarded fundamental solutions for $\Box=\delta\mathrm{d}+\mathrm{d}\delta$ acting on $k$-forms in ultrastatic spacetimes with timelike boundary. We used the technique of {boundary triples}, imposing suitable classes of boundary conditions, dubbed $\delta\mathrm{d}$-tangential and $\delta\mathrm{d}$-normal. Subsequently, we applied these results identifying the space of solutions of $\delta\mathrm{d} A=0$ under $\delta\mathrm{d}$-tangential and $\delta\mathrm{d}$-normal boundary conditions, distinguishing two different notions of gauge invariance and showing that within the two gauge equivalence classes it is always possible to find a representative that abides to Lorenz gauge. We identified, for the two boundary conditions, the optimal algebras of observables, showing that in general they do possess a non-trivial centre.\\

\vskip1cm


Possible extensions to this work arise for both the formulations of Maxwell's equations in terms of $F\in\Omega^2(M)$ and in terms of $A\in\Omega^k(M)$.\vskip0.25cm

In particular, focusing on the results of Chapter \ref{Chapter2}, in the discussion about the non-dynamical part of Maxwell's equations for $F$ (\emph{cf.} Section \ref{Sec: Non-dynamical equations: Hodge theory with interface}), we assumed that the underlying Cauchy hypersurface was closed, i.e. compact and with no boundary, in order to use Hodge decomposition. Hodge decomposition was used to provide a mathematically rigorous framework for the non-dynamical equations. It is natural to ask whether a generalization of Hodge decomposition can be used on a larger class of Cauchy hypersurfaces (\emph{cf.} Subsection \ref{Sec: weak-Hodge decomposition}).

Moreover, we did not construct explicitly the algebra of observables for the Faraday tensor $F$, while we only gave an account on the possible strategy in Section \ref{Sec: perspectives on F}.\\

On the other side, the results of Chapter \ref{Chapter3} can be generalized in various directions. At first, we considered the wave operator $\Box$ and we indicated a class of boundary conditions, encoded by $\Omega^k_{f,f'}(M)$ which ensured the operator to be closed or, in other words, formally self-adjoint. It is important to stress that the boundary conditions $\Omega^k_{f,f'}(M)$ are not the largest class which makes the operator closed. As a matter of fact one can think of other possibilities, for example those similar to the Wentzell boundary conditions, which were studied in the scalar
scenario in \cite{Dappiaggi-Drago-Ferreira-19,Dappiaggi:2018pju,Zahn:2015due} and that could be interesting from a physical viewpoint. Therefore it arises the question whether there is a larger class of boundary conditions for $\Box$ such that the existence of fundamental solutions can be established.

Subsequently, in Subsection \ref{Sub: properties of G}, we assumed the existence of distinguished fundamental solutions for the wave operator and studied their properties, which remain valid whenever the hypothesis of Assumption \ref{Thm: assumption theorem} hold. To show that the Assumption can be verified, we used a particular technique from functional analysis which is well-suited for static spacetimes. Hence, a possible follow-up is to investigate whether there are other methods that allow the construction of fundamental solutions for $\Box$ in non-ultrastatic spacetimes.\thispagestyle{plain}

Regarding the construction of the spaces of solutions for Maxwell's equations $\delta\mathrm{d}A=0$, we restricted the discussion to a particular pair of boundary conditions, $\delta\mathrm{d}$-tangential and $\delta\mathrm{d}$-normal, for which it was possible to find a gauge-equivalent solution which satisfied the wave equation $\Box A=0$.
It may be interesting to investigate whether there are other boundary conditions for the Maxwell operator $\delta\mathrm{d}$ such that the construction of fundamental solutions is possible without resorting to the Lorenz gauge and thus relying on the fundamental solutions for $\Box$ -- \emph{cf.} Section \ref{Sec: on the Maxwell operator}.\thispagestyle{plain}

To conclude, we think that the work should be recast regarding electromagnetism as a Yang-Mills Abelian theory for the connections on a $U(1)$-principal bundle, rather than choosing $\mathbb{R}$ as a structure group. This should affect the choice of the gauge groups and hence the formulation of the spaces of solutions and of the algebra of observables -- \emph{cf.} the discussion in Subsection \ref{Sub: solutions maxwell}.\thispagestyle{plain}










