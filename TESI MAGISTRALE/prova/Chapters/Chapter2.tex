\chapter{Maxwell equations with interface conditions} % Main chapter title

\label{Chapter2} % For referencing the chapter elsewhere, use \ref{Chapter1}



%Section \ref{Sec: Maxwell equations with interface conditions} deals with the construction of the algebra of observable associated with the Faraday tensor $F$ in the presence of an interface $Z$.
%For that we need Hodge decomposition, boundary triples out of which we define an exact sequence -- \textit{cf.} proposition \ref{Prop: exact sequence for Maxwell equations with interface}.

As outlined in Section \ref{Sec: Maxwell introduction}, the very nature of Maxwell equations allows us to use both $F$ and $A$ as variables with which to describe electromagnetic phenomena. Whenever the second cohomology group $H^2(M)$ is trivial, the two theories are equivalent, since $F=\mathrm{d}A$.


In this chapter, we regard $F\in\Omega^2(M)$ as the true physical dynamical variable which describes electromagnetism. The aim of this chapter is to present a technique which allows to characterize, in a class of manifolds with the presence of an interface between two media, the existence of fundamental solutions for Maxwell equations, written in terms of the Faraday form $F\in\Omega^2(M)$. The presence of an interface on the one hand generalizes the idea of the presence of a timelike boundary, allowing to recover the geometric setting outlined in Chapter \ref{Chapter1} in case on one side of the interface lies a perfect insulator. On the other hand, in order to make use of geometric techniques such as Hodge decomposition, we will have to make several geometric assumptions which ensure global hyperbolicity, but unfortunately are way less general.

\section{Geometrical set-up}
The physical and practical situation we want to approach is that of a manifold split into two parts, filled with two media, each of them with different electromagnetic properties. The two media will be separated by an hypersurface, on which our aim will be that of putting \emph{jump conditions}.

We consider a static Lorentzian manifold $(M,g)$ with \textbf{empty boundary}, such that $M$ can be decomposed as $\mathbb{R}\times\Sigma$, where the Cauchy hypersurface $(\Sigma,h)$ is assumed to be a complete, connected, odd-dimensional, \textbf{closed} Riemannian manifold.
%\\\nicomment{(We should look for references where the (weak) Hodge decomposition is established for non-compact Riemannian manifold with boundary. If this is the case and if the results presented below remain valid with the weak Hodge decomposition, we will drop the closedness assumption.)}\\
The assumptions we made so far on imply that $(M,g)$ is a globally hyperbolic spacetime without boundary.

Maxwell equations, recalling formulas \eqref{Eqn: first maxwell} and \eqref{Eqn: second maxwell}, for $F\in\Omega^2(M)$ are simply
\begin{align}\label{Eqn: covariant Maxwell equations}
\mathrm{d}F=0\,,\qquad\delta F=0\,,
\end{align}
The geometrical assumptions on $M$ permit us to split $F$ into electric and magnetic components
\begin{align}\label{Eqn: electric and magnetic components}
F=\ast_\Sigma B+\mathrm{d}t\wedge E\,,
\end{align}
where $E,B\in\Omega^1(\Sigma)$ while $\ast_\Sigma$ is the Hodge dual of $\Sigma$.\\
Maxwell equations are then reduced to
\begin{subequations}\label{Eqn: Maxwell equations}
	\begin{align}
	\label{Eqn: dynamical Maxwell equations}
	\partial_tE-\mathrm{curl}B=0\,,\qquad
	\partial_tB+\mathrm{curl}E=0\,,\\
	\label{Eqn: non-dynamical Maxwell equations}
	\mathrm{div}(E)=\mathrm{div}(B)=0\,,
	\end{align}
\end{subequations}
where $\mathrm{div}=\delta_\Sigma$ is the co-differential of $\Sigma$, while $\mathrm{curl}$ is defined in equation \eqref{Eqn: curl convention} -- in particular $\mathrm{curl}=\ast_\Sigma\mathrm{d}_\Sigma$ if $\dim\Sigma=3\mod 4$.\\
To model the presence of an interface that divides $M$ in two distinct regions, we also let $Z$ be a codimension $1$ smooth embedded hypersurface of $\Sigma$.

We denote with $\mathrm{d},\delta$ the differential and co-differential over $M$, while $\mathrm{d}_\Sigma,\delta_\Sigma$ denote the differential and co-differential over $\Sigma$.

In this setting we would like to consider Maxwell equations with $Z$-interface boundary conditions. This means that we will consider Maxwell equations on $M\setminus(\mathbb{R}\times Z)$, allowing for jump discontinuities to occur on $\mathbb{R}\times Z$.
\\\nicomment{(We should decide whether we want to consider $k$-Maxwell equations or not.
	In the former case the $\mathrm{curl}$ operator acts as $\mathrm{curl}\colon\Omega^k(\Sigma)\to\Omega^k(\Sigma)$ where $\dim\Sigma=2k+1$.)}\\


Whenever $Z\neq\emptyset$ the system \eqref{Eqn: Maxwell equations} has to be modified, in particular the non-dynamical equations \eqref{Eqn: non-dynamical Maxwell equations} involving the divergence operator $\mathrm{div}$ have to be suitably interpreted -- \textit{cf.} remark \ref{Rmk: Hodge formulation of non-dynamical Maxwell equations}.
In particular one expects that the condition $\operatorname{div}(E)=\mathrm{div}(B)=0$ should be interpreted weakly, leading to a constraint on the normal jump of $E$ across $Z$.
Moreover, the dynamical equations \eqref{Eqn: dynamical Maxwell equations} have to be combined with boundary conditions at the interface $Z$ -- \textit{cf.} \cite[Sec. I.5]{Jackson-99}.

In what follows we will state the precise meaning of the problem \eqref{Eqn: Maxwell equations} with interface $Z$ with the help of Hodge theory and Lagrangian subspaces \cite{Everitt-Markus-99,Everitt-Markus-03,Everitt-Markus-05}.
%boundary triples theory \cite{Behrndt-Langer-12}.


\section{Non-dynamical equations: Hodge theory with interface}\label{Sec: Non-dynamical equations: Hodge theory with interface}
In this section we present a Hodge decomposition for the closed Riemannian manifold $(\Sigma,h)$ with interface $Z$.
This generalizes the known results on classical Hodge decomposition on manifolds with possible non-empty boundary \cite{Amar-17,Axelsson-McIntosh-04,Gaffney-55,Gromov-91,Kodaira-49,Li-09,Schwarz-95,Scott-95,Zulfikar-Stroock-00}.\\



Hodge theory comes as a generalization of Helmholtz decomposition. He first formulated a result on the splitting of vector fields into vortices and gradients, which can be understood as a rudimentary form of what is now called the \emph{Hodge decomposition}. The idea behind Helmholtz decomposition is that any vector field in $\mathbb{R}^3$ can be decomposed as a sum of an irrotational field, i.e. $\operatorname{curl}=\mathrm{d}_\Sigma=0$, and a solenoidal field, i.e. $\operatorname{div}=\delta_\Sigma=0$. In other words, for $\mathbf{F}\in C^2(\mathbb{R}^3,\mathbb{R}^3)$, one can write
\begin{equation}
\mathbf{F}=-\nabla\Phi+\operatorname{curl}\mathbf{A}.
\end{equation}


In what follows $\mathrm{L}^2\Omega^k(\Sigma)$ will denote the space of sections of $\wedge^kT^*\Sigma$ which are square integrable with respect to the pairing induced by the metric $h$
\begin{align}\label{Eqn: L2-scalar product}
	(\alpha,\beta)_\Sigma:=\int_\Sigma\overline{\alpha}\wedge\ast_\Sigma\beta\,,
\end{align}
where $\ast_\Sigma$ is the Hodge dual.
Similarly we shall denote with $C^\infty_{\mathrm{c}}\Omega^k(\Sigma)$ the space of smooth and compactly supported $k$-forms, while $\mathrm{H}^\ell\Omega^k(\Sigma)$ will denote $k$-forms with weak $\mathrm{L}^2$-derivatives up to order $\ell\in\mathbb{N}\cup\{0\}$ with respect to one (hence all) connection over $\Sigma$ -- as usual we also set $\mathrm{H}^{-\ell}\Omega^k(\Sigma):=\mathrm{H}^\ell\Omega^k(\Sigma)^*$.
\\\nicomment{(If $\Sigma$ is not compact we only have that only $\mathrm{H}^\ell_{\mathrm{loc}}(\Sigma)$ is independent from the choice of the connection.
	If we assume $(\Sigma,h)$ to be of bounded geometry the ambiguity disappears because of the results of \cite{Eichorn-93}.)}\\

\subsection{Hodge decomposition on compact manifold with non-empty boundary}
The Hodge theorem for a closed manifold $\Sigma$ states that there is an $\mathrm{L}^2$-orthogonal decomposition
\begin{align}\label{Eqn: Hodge decomposition on closed manifolds}
	\mathrm{L}^2\Omega^k(\Sigma)=\mathrm{d}_\Sigma \mathrm{H}^1\Omega^{k-1}(\Sigma)\oplus\delta_\Sigma \mathrm{H}^1\Omega^{k+1}(\Sigma)\oplus\ker(\Delta)_{\mathrm{H}^1\Omega^k(\Sigma)}\,,
\end{align}
where $\Delta=\mathrm{d}_\Sigma\delta_\Sigma+\delta_\Sigma\mathrm{d}_\Sigma$ is the Laplacian and $\ker(\Delta)_{\mathrm{H}^1\Omega^k(\Sigma)}$ denotes the space of \emph{harmonic forms}.
If $\Sigma$ has no an empty boundary, the space of harmonic forms $\ker(\Delta)_{\mathrm{H}^1\Omega^k(\Sigma)}$ coincides with that of \textbf{harmonic fields}, defined (following \cite{Kodaira-49} and \cite{Schwarz-95}) as
\begin{equation}\label{Eqn: harmonic fields}
	\mathcal{H}^k(\Sigma)=\lbrace\omega\in \mathrm{H}^1\Omega^k(\Sigma)|\;\mathrm{d}_\Sigma\omega=0\,,\;\delta_\Sigma\omega=0\rbrace=\ker(\delta_\Sigma)_{\mathrm{H}^1\Omega^k(\Sigma)}\cap\ker(\mathrm{d}_\Sigma)_{\mathrm{H}^1\Omega^k(\Sigma)}\,.
\end{equation}
The last result can be stated as follows and it is very easy to prove.
\begin{proposition}
	Let $\alpha\in\mathrm{H}^1\Omega^k(\Sigma)$, where $\Sigma$ is a closed manifold. Then $\Delta\alpha=0$ if and only if $\mathrm{d}_\Sigma\alpha=0$ and $\delta_\Sigma\alpha=0$.
\end{proposition}
\begin{proof}
	Clearly if $\mathrm{d}_\Sigma\alpha=0$ and $\delta_\Sigma\alpha=0$, $\Delta\alpha=0$. On the other hand if $\Delta\alpha=0$,
	\begin{align}
		0=&\left(\Delta\alpha,\alpha\right)_\Sigma=\left((\mathrm{d}_\Sigma\delta_\Sigma+\delta_\Sigma\mathrm{d}_\Sigma)\alpha,\alpha\right)_\Sigma=\left(\mathrm{d}_\Sigma\delta_\Sigma\alpha,\alpha\right)_\Sigma+\left(\delta_\Sigma\mathrm{d}_\Sigma\alpha,\alpha\right)_\Sigma=\\
		=&\left(\delta_\Sigma\alpha,\delta_\Sigma\alpha\right)_\Sigma+\left(\mathrm{d}_\Sigma\alpha,\mathrm{d}_\Sigma\alpha\right)_\Sigma=||\delta_\Sigma\alpha||^2+||\mathrm{d}_\Sigma\alpha||^2.
	\end{align}
	So both $\mathrm{d}_\Sigma\alpha=0$ and $\delta_\Sigma\alpha=0$.
\end{proof}
For a compact manifold $\Sigma$ with non-empty boundary $\partial\Sigma$ the decomposition \eqref{Eqn: Hodge decomposition on closed manifolds} requires a slight adjustment and harmonic forms do not coincide with harmonic fields anymore.
Because of boundary terms, $\ker\Delta$ no longer coincides with the closed and co-closed forms. It is clear that every harmonic field is a harmonic form, but the converse is false. To show this, consider the following example.
\begin{Example}
	Let $U$ a bounded subset of $\mathbb{R}^2$, endowed with the standard euclidean metric. On $U$, the 1-form $\omega=x \,dy$ is clearly harmonic, since its second derivatives vanish, but it is not in $\ker \mathrm{d}$ as
	
	\[\mathrm{d}(x\, \mathrm{d}y) = \partial_x\, x\, \mathrm{d}x \wedge \mathrm{d}y + \partial_y\, x\, \mathrm{d}y \wedge \mathrm{d}y = \mathrm{d}x \wedge \mathrm{d}y.\] $\omega$ is though in $\ker \delta$ as $\ast \mathrm{d} \ast (x\, \mathrm{d}y) =\ast \mathrm{d}(x \,\mathrm{d}x) = 0$.
\end{Example}


In fact, the space $\mathcal{H}^k(\Sigma)$ of harmonic fields is infinite dimensional and so is much too big to represent the cohomology, and to recover the Hodge isomorphism one has to impose boundary conditions. Indeed the spaces $\mathrm{d}_\Sigma \mathrm{H}^1\Omega^{k-1}(\Sigma)$, $\delta_\Sigma H^{1}\Omega^{k+1}(\Sigma)$, $\mathcal{H}^k(\Sigma)$ are not orthogonal unless suitable boundary conditions are imposed.

\begin{remark}\label{Rmk: extension of tangential and normal maps to Sobolev spaces}
	According to \cite[p. 171]{Georgescu-79}, the tangential and normal maps defined in Definition \ref{Def: tangential and normal component} can be extended to continuous surjective maps
	\begin{align}\label{Eqn: Sobolev tangential and normal trace maps}
		\mathrm{t}\oplus\mathrm{n}\colon
		\mathrm{H}^\ell\Omega^k(\Sigma)\to
		\mathrm{H}^{\ell-\frac{1}{2}}\Omega^k(\partial\Sigma)\oplus
		\mathrm{H}^{\ell-\frac{1}{2}}\Omega^k(\partial\Sigma)\,\qquad\forall\ell\geq\frac{1}{2}\,.
	\end{align}
\end{remark}
We can now recall the Hodge decomposition for compact manifolds with boundary \cite[Thm. 2.4.2]{Schwarz-95}.
\begin{theorem}\label{Thm: Hodge decomposition for manifolds with boundary}
	Let $(\Sigma,h)$ be a compact, connected, Riemannian manifold with non-empty boundary $\partial\Sigma\stackrel{\iota_{\partial\Sigma}}{\hookrightarrow}\Sigma$.
	\begin{enumerate}
		\item
		For all $\omega\in C^{\infty}_{\mathrm{c}}\Omega^{k-1}(\Sigma)$ and $\eta\in C^{\infty}_{\mathrm{c}}\Omega^{k}(\Sigma)$ it holds
		\begin{align}\label{Eqn: boundary terms}
			(\mathrm{d}_\Sigma\omega,\eta)_\Sigma-(\omega,\delta_\Sigma\eta)_\Sigma=
			(\mathrm{t}\omega,\mathrm{n}\eta)_{\partial\Sigma}\,,
%			\int_{\partial\Sigma}\mathrm{t}\overline{\omega}\wedge\ast_{\Sigma}\mathrm{n}\eta\,,
		\end{align}
		where $(\;,\;)_\Sigma$ has been defined in equation \eqref{Eqn: L2-scalar product} while $(\;,\;)_{\partial\Sigma}$ is defined similarly.
		Equation \eqref{Eqn: boundary terms} still holds true for $\omega\in \mathrm{H}^\ell\Omega^{k-1}(\Sigma)$ and $\eta\in \mathrm{H}^\ell\Omega^{k}(\Sigma)$ -- \textit{cf.} remark \ref{Rmk: extension of tangential and normal maps to Sobolev spaces}.
		\item
		There is an orthogonal decomposition
		\begin{align}\label{Eqn: Hodge decomposition for manifold with boundary}
			\mathrm{L}^2\Omega^k(\Sigma)=
			\mathrm{d}_\Sigma \mathrm{H}^1\Omega^k_{\mathrm{t}}(\Sigma)\oplus
			\delta_\Sigma \mathrm{H}^1\Omega^{k+1}_{\mathrm{n}}(\Sigma)
			\oplus\mathrm{L}^2\mathcal{H}^k(\Sigma)\,,		
		\end{align}
		where $\mathrm{L}^2\mathcal{H}^k(\Sigma)$ is the closure with respect to the $\mathrm{L}^2$ norm of the space of harmonic fields, as defined per equation \eqref{Eqn: harmonic fields} and
		\begin{align}\label{Eqn: Dirichlet and Neumann forms}
			\mathrm{H}^1\Omega^{k-1}_{\mathrm{t}}(\Sigma):=\lbrace\alpha\in \mathrm{H}^1\Omega^{k-1}(\Sigma)|\;\mathrm{t}\alpha=0\rbrace\,,\\
			\mathrm{H}^1\Omega^{k+1}_{\mathrm{n}}(\Sigma):=\lbrace\beta\in \mathrm{H}^1\Omega^{k+1}(\Sigma)|\;\mathrm{n}\beta=0\rbrace\,,
		\end{align}
	following the definitions of Equation \eqref{Eqn: k-forms with vanishing tangential or normal component}.
	\end{enumerate}
\end{theorem}
\begin{remark}
		The previous decomposition generalizes to Sobolev spaces, in particular for all $\ell\in\mathbb{N}\cup\{0\}$ we have
		\begin{align}\label{Eqn: Hodge decomposition for manifold with boundary for Sobolev spaces}
			\mathrm{H}^\ell\Omega^k(\Sigma)=\mathrm{d}_\Sigma \mathrm{H}^{\ell+1}\Omega^k_{\mathrm{t}}(\Sigma)\oplus\delta_\Sigma \mathrm{H}^{\ell+1}\Omega^{k+1}_{\mathrm{n}}\oplus
			\mathrm{H}^\ell \mathcal{H}^k(\Sigma)\,,		
		\end{align}
		where $\mathrm{H}^\ell \mathcal{H}^k(\Sigma)=\mathrm{L}^2\mathcal{H}^k(\Sigma)\cap\mathrm{H}^\ell\Omega^k(\Sigma)$, since $\mathrm{H}^\ell\Omega^k(\Sigma)\hookrightarrow\mathrm{L}^2\Omega^k(\Sigma)$.
\end{remark}


\subsubsection{Hodge decomposition for compact manifold with interface}
We would like to consider a decomposition similar to the one of theorem \ref{Thm: Hodge decomposition for manifolds with boundary} for the case of a closed Riemannian manifold $\Sigma$ with interface $Z$.
In this setting we split $\Sigma_Z:=\Sigma\setminus Z=\Sigma_+\cup \Sigma_-$ and we refer to $\Sigma_-$ (\textit{resp}. $\Sigma_+$) as the left (\textit{resp}. right) component of $\Sigma$.
Notice that $\Sigma_\pm$ are compact manifolds with boundary $\partial \Sigma_\pm=\pm Z$ -- notice that the orientation on $Z$ induced by $\Sigma_+$ is the opposite of the one induced by $\Sigma_-$.
Therefore theorem \ref{Thm: Hodge decomposition for manifolds with boundary} applies to $\mathrm{L}^2\Omega^k(\Sigma_\pm)$.

Since $Z$ has zero measure the space of square integrable $k$-forms splits as
\begin{align}\label{Eqn: splitting of k-forms with interface}
	\mathrm{L}^2\Omega^k(\Sigma)=
	\mathrm{L}^2\Omega^k(\Sigma_Z)=
	\mathrm{L}^2\Omega^k(\Sigma_+)\oplus\mathrm{L}^2\Omega^k(\Sigma_-)\,.
\end{align}
We expect a $Z$-relative Hodge decomposition as in \eqref{Eqn: Hodge decomposition for manifold with boundary} to hold true in this situation, where the boundary conditions of the spaces $\mathrm{H}^1\Omega^{k-1}_{\mathrm{t}}(\Sigma)$, $\mathrm{H}^1\Omega^{k-1}_{\mathrm{n}}(\Sigma)$ should be replaced by appropriate jump conditions across $Z$.
For that, notice that the splitting \eqref{Eqn: splitting of k-forms with interface} does not generalize to the Sobolev spaces $\mathrm{H}^\ell\Omega^k(\Sigma)$, in particular
\begin{align}
	\mathrm{H}^\ell\Omega^k(\Sigma)\subset
	\mathrm{H}^\ell\Omega^k(\Sigma_Z)=
	\mathrm{H}^\ell\Omega^k(\Sigma_+)\oplus\mathrm{H}^\ell\Omega^k(\Sigma_-)\,,
\end{align}
is a proper inclusion.
\\\nicomment{(Is it clear that the objects associated with $\Sigma_Z$ are automatically a direct sum of objects associated with $\Sigma_\pm$?)}

\begin{Definition}\label{Def: jump of tangential and normal component}
	Let $(\Sigma,h)$ be an oriented, compact, Riemanniann manifold with interface $Z\stackrel{\iota_Z}{\hookrightarrow}\Sigma$.
	Moreover let $(\Sigma_\pm,h_\pm)$ the oriented, compact Riemannian manifolds with boundary $\partial\Sigma_\pm=\pm Z$ such that $\Sigma_Z:=
	\Sigma\setminus Z=\Sigma_+\cup\Sigma_-$.
	For $\omega\in C^\infty\Omega^k(\Sigma_Z)$ we define the tangential jump $[\mathrm{t}\omega]\in C^\infty\Omega^k(Z)$ and normal jump $[\mathrm{n}\omega]\in C^\infty\Omega^{k-1}(Z)$ across $Z$ by
	\begin{align}\label{Eqn: tangential and normal jump}
		[\mathrm{t}\omega]:=\mathrm{t}_+\omega-\mathrm{t}_-\omega\,,\qquad
		[\mathrm{n}\omega]:=\mathrm{n}_+\omega-\mathrm{n}_-\omega\,,
	\end{align}
	where $\mathrm{t}_\pm$, $\mathrm{n}_\pm$ denote the tangential and normal map on $\Sigma_\pm$ as per definition \ref{Def: tangential and normal component}.
\end{Definition}
\begin{remark}\label{Rmk: spaces with no jumps}
	It is an immediate consequence of definition \ref{Def: jump of tangential and normal component} that
	\begin{align}
		\mathrm{H}^1\Omega^k(\Sigma)=
		\lbrace\omega\in\mathrm{H}^1\Omega^k(\Sigma_Z)|\;[\mathrm{t}\omega]=0\,,\;[\mathrm{n}\omega]=0
		\rbrace\,.
	\end{align}
	The same equality does not hold for $C^\infty\Omega^k(\Sigma)$ because higher order traces have to match at $Z$.
\end{remark}
\begin{theorem}\label{Thm: Hodge decomposition for manifolds with interface}
	Let $(\Sigma,h)$ be an oriented, compact, Riemanniann manifold with interface $Z\stackrel{\iota_Z}{\hookrightarrow}\Sigma$.
	Moreover let $(\Sigma_\pm,h_\pm)$ the oriented, compact Riemannian manifolds with boundary $\partial\Sigma_\pm=\pm Z$ such that $\Sigma\setminus Z=\Sigma_+\cup\Sigma_-$.
	\begin{enumerate}
		\item 
		For all $\omega\in C^{\infty}_{\mathrm{c}}\Omega^{k-1}(\Sigma_Z)$ and $\eta\in C^{\infty}_{\mathrm{c}}\Omega^k(\Sigma_Z)$ it holds
		\begin{align}\label{Eqn: boundary terms with interface}
			(\mathrm{d}_\Sigma\omega,\eta)_Z-(\omega,\delta_\Sigma\eta)_Z=
			([\mathrm{t}\omega],\mathrm{n}_+\eta)_Z-(\mathrm{t}_-\omega,[\mathrm{n}\eta])_Z\,,
%			\int_Z[\mathrm{t}\overline{\omega}]\wedge\ast_{\Sigma}\mathrm{n}_+\eta-
%			\int_Z\mathrm{t}_-\overline{\omega}\wedge\ast[\mathrm{n}\eta]\,,
		\end{align}
		where $(\;,\;)_Z$ is the scalar product between forms on $Z$ -- \textit{cf}. equation \eqref{Eqn: L2-scalar product} -- while $\mathrm{t}_\pm$, $\mathrm{n}_\pm$ are the tangential and normal maps on $\Sigma_\pm$ as per definition \ref{Def: tangential and normal component}.
		Equation \eqref{Eqn: boundary terms} still holds true for $\omega\in \mathrm{H}^\ell\Omega^{k-1}(\Sigma)$ and $\eta\in \mathrm{H}^\ell\Omega^k(\Sigma)$ for all $\ell\geq 1$.
		\item
		There is an orthogonal decomposition
		\begin{align}\label{Eqn: Hodge decomposition for manifold with interface}
			\mathrm{L}^2\Omega^k(\Sigma)=
			\mathrm{d}_\Sigma \mathrm{H}^1\Omega^k_{[\mathrm{t}]}(\Sigma_Z)\oplus
			\delta_\Sigma \mathrm{H}^1\Omega^{k+1}_{[\mathrm{n}]}(\Sigma_Z)\oplus\mathcal{H}^k(\Sigma)\,,		
		\end{align}
		where we defined $\mathcal{H}^k(\Sigma)$ as per equation \eqref{Eqn: harmonic fields} and
		\begin{align}\label{Eqn: Dirichlet and Neumann jump forms}
			\mathrm{H}^1\Omega^{k-1}_{[\mathrm{t}]}(\Sigma):=\lbrace\alpha\in \mathrm{H}^1\Omega^{k-1}(\Sigma_Z)|\;[\mathrm{t}\alpha]=0\rbrace\,,\\
			\mathrm{H}^1\Omega^{k+1}_{[\mathrm{n}]}(\Sigma):=\lbrace\beta\in \mathrm{H}^1\Omega^{k+1}(\Sigma_Z)|\;[\mathrm{n}\beta]=0\rbrace\,.
		\end{align}
	\end{enumerate}
\end{theorem}
\begin{proof}
	Equation \eqref{Eqn: boundary terms with interface} is an immediate consequence of \eqref{Eqn: boundary terms}.
	In particular for $\omega\in C^{\infty}_{\mathrm{c}}\Omega^{k-1}(\Sigma_Z)$ and $\eta\in C^{\infty}_{\mathrm{c}}\Omega^k(\Sigma_Z)$ we decompose $\omega=\omega_++\omega_-$ and $\eta=\eta_++\eta_-$ where $\omega_\pm\in C^\infty_{\mathrm{c}}\Omega^{k-1}(\Sigma_\pm)$ and $\eta_\pm\in C^\infty_{\mathrm{c}}\Omega^k(\Sigma_\pm)$.
	(Notice that with this notation we have $\mathrm{t}_\pm\omega=\mathrm{t}_\pm\omega_\pm$.)
	Applying equation \eqref{Eqn: boundary terms} we have
	\begin{align*}
		(\mathrm{d}_\Sigma\omega,\eta)-(\omega,\delta_\Sigma\eta)&=
		\sum_\pm\big((\mathrm{d}_\Sigma\omega_\pm,\eta_\pm)-(\omega_\pm,\delta_\Sigma\eta_\pm)\big)=
		\int_Z\mathrm{t}_+\overline{\omega}\wedge\ast_\Sigma\mathrm{n}_+\eta-
		\int_Z\mathrm{t}_-\overline{\omega}\wedge\ast_\Sigma\mathrm{n}_-\eta\\&=
		\int_Z[\mathrm{t}\overline{\omega}]\wedge\ast_{\Sigma}\mathrm{n}_+\eta-
		\int_Z\mathrm{t}_-\overline{\omega}\wedge\ast[\mathrm{n}\beta]\,.
	\end{align*}
	A density argument leads to the same identity for $\omega\in\mathrm{H}^\ell\Omega^{k-1}(\Sigma_Z)$ and $\eta\in\mathrm{H}^{\ell}\Omega^k(\Sigma_Z)$ for $\ell\geq 1$.
	\\
	We now prove the splitting \eqref{Eqn: Hodge decomposition for manifold with interface}.
	The spaces $\mathrm{d}_\Sigma \mathrm{H}^1\Omega^k_{[\mathrm{t}]}(\Sigma_Z)$, $\delta_\Sigma \mathrm{H}^1\Omega^{k+1}_{[\mathrm{n}]}(\Sigma_Z)$, $\mathcal{H}^k(\Sigma)$ are orthogonal because of equation \eqref{Eqn: boundary terms with interface}.
	Let now $\omega$ be in the orthogonal complement of $\mathrm{d}_\Sigma \mathrm{H}^1\Omega^k_{[\mathrm{t}]}(\Sigma_Z)\oplus\delta_\Sigma \mathrm{H}^1\Omega^{k+1}_{[\mathrm{n}]}(\Sigma_Z)$.
	We wish to show that $\omega\in\mathcal{H}^k(\Sigma)$.
	We split $\omega=\omega_++\omega_-$ with $\omega_\pm\in\mathrm{L}^2\Omega^k(\Sigma_\pm)$ and apply theorem \ref{Thm: Hodge decomposition for manifolds with boundary} to each component so that
	\begin{align*}
		\omega=
		\sum_\pm\big(\mathrm{d}_\Sigma\alpha_\pm+\delta_\Sigma\beta_\pm+\kappa_\pm\big)\,,
	\end{align*}
	where $\alpha_\pm\in\mathrm{H}^1\Omega^{k-1}_{\mathrm{t}}(\Sigma_\pm)$, $\beta_\pm\in\mathrm{H}^1\Omega^{k+1}_{\mathrm{n}}(\Sigma_\pm)$ and $\kappa_\pm\in\mathcal{H}^k(\Sigma_\pm)$.
	Let now be $\hat{\alpha}\in\mathrm{H}^1\Omega^{k-1}(\Sigma_+)$: this defines an element in $\Omega^{k-1}_{[\mathrm{t}]}(\Sigma_Z)$ by considering its extension by zero on $\Sigma_-$.
	Since $\omega\perp\mathrm{d}_\Sigma\mathrm{H}^1\Omega_{[\mathrm{t}]}(\Sigma_Z)$ we have $0=(\mathrm{d}_\Sigma\hat{\alpha},\omega)=(\mathrm{d}_\Sigma\hat{\alpha},\mathrm{d}_\Sigma\alpha_+)$, thus $\mathrm{d}_\Sigma\alpha_+=0$ by the arbitrariness of $\hat{\alpha}$.
	With a similar argument we have $\alpha_-=0$ as well as $\beta_\pm=0$.
	\\
	Therefore $\omega\in\mathcal{H}^k(\Sigma_Z)$.
	In order to prove that $\omega\in\mathcal{H}^k(\Sigma)$ we need to show that $[\mathrm{t}\omega]=0$ as well as $[\mathrm{n}\omega]=0$ -- \textit{cf.} remark \ref{Rmk: spaces with no jumps}.
	This is a consequence of $\omega\perp\mathrm{d}_\Sigma \mathrm{H}^1\Omega^k_{[\mathrm{t}]}(\Sigma_Z)\oplus\delta_\Sigma \mathrm{H}^1\Omega^{k+1}_{[\mathrm{n}]}(\Sigma_Z)$.
	Indeed, let $\alpha\in\mathrm{H}^1\Omega^{k-1}_{[\mathrm{t}]}(\Sigma_Z)$: applying equation \eqref{Eqn: boundary terms with interface} we find
	\begin{align}
		0=(\mathrm{d}_\Sigma\alpha,\omega)=-\int_Z\mathrm{t}_-\overline{\alpha}\wedge\ast[\mathrm{n}\omega]\,.
	\end{align}
	The arbitrariness of $\mathrm{t}_-\alpha$ implies $[\mathrm{n}\omega]=0$.
	Similarly $[\mathrm{t}\omega]=0$ follows by $\omega\perp\delta_\Sigma\mathrm{H}^1\Omega^{k+1}_{[\mathrm{n}]}(\Sigma_Z)$.
\end{proof}
\begin{remark}
	The harmonic part of decomposition \eqref{Eqn: Hodge decomposition for manifold with interface} contains harmonic $k$-forms which are continuous across the interface $Z$ -- \textit{cf.} remark \ref{Rmk: spaces with no jumps}.
	One can also consider a decomposition which allows for a discontinuous harmonic component: in particular it can be shown that
	\begin{align*}
		\mathrm{L}^2\Omega^k(\Sigma)=
		\mathrm{d}_\Sigma\mathrm{H}^1\Omega^{k-1}_{\mathrm{t}}(\Sigma_Z)\oplus
		\delta_\Sigma\mathrm{H}^1\Omega^{k+1}_{\mathrm{n}}(\Sigma_Z)\oplus
		\mathcal{H}^k(\Sigma_Z)\,,
	\end{align*}
	where now $\mathrm{H}^1\Omega^{k-1}_{\mathrm{t}}(\Sigma_Z)$ is the subspace of $\mathrm{H}^1\Omega^{k-1}_{[\mathrm{t}]}(\Sigma_Z)$ made of $(k-1)$-forms $\alpha$ such that $\mathrm{t}_\pm\omega=0$ and similarly $\beta\in\mathrm{H}^1\Omega^{k+1}_{\mathrm{n}}(\Sigma_Z)$ if and only if $\beta\in\mathrm{H}^1\Omega^{k+1}_{[\mathrm{n}]}(\Sigma_Z)$ and $\mathrm{n}_\pm\beta=0$.
\end{remark}
\begin{remark}\label{Rmk: weak-Hodge decomposition}
	The results of theorem \ref{Thm: Hodge decomposition for manifolds with boundary} generalize in several directions \cite{Amar-17,Axelsson-McIntosh-04,Gaffney-55,Gromov-91,Kodaira-49,Li-09,Schwarz-95,Scott-95,Zulfikar-Stroock-00}.
	For the case of a non-compact Riemannian manifold $\Sigma$ one may follow the results of \cite{Axelsson-McIntosh-04} in order to achieve the following weak-Hodge decomposition -- \textit{cf}. equation \eqref{Eqn: Hodge decomposition for manifold with boundary}.
	We consider the operators $\mathrm{d}_{\Sigma,\mathrm{t}},\delta_{\Sigma,\mathrm{n}}$ defined by
	\begin{align}
		\label{Eqn: Dirichlet differential}
		\operatorname{dom}(\mathrm{d}_{\Sigma,\mathrm{t}})&:=\lbrace
		\omega\in\mathrm{L}^2\Omega^k(\Sigma)|\;\mathrm{d}_\Sigma\omega\in\mathrm{L}^2\Omega^{k+1}(\Sigma)\,,\;\mathrm{t}\omega=0\rbrace\qquad
		\mathrm{d}_{\Sigma,\mathrm{t}}\omega:=\mathrm{d}_\Sigma\omega\,,\\
		\label{Eqn: Neumann codifferential}
		\operatorname{dom}(\delta_{\Sigma,\mathrm{n}})&:=\lbrace
		\omega\in\mathrm{L}^2\Omega^k(\Sigma)|\;\delta_\Sigma\omega\in\mathrm{L}^2\Omega^{k-1}(\Sigma)\,,\;\mathrm{n}\omega=0\rbrace\qquad
		\delta_{\Sigma,\mathrm{n}}\omega:=\delta_\Sigma\omega\,.
	\end{align}
	Notice that $\mathrm{d}_{\Sigma,\mathrm{t}}$ as well as $\delta_{\Sigma,\mathrm{n}}$ are nihilpotent because of relations \eqref{Eqn: relation between tangential and normal component with Hodge operator, diff and codiff}.
	These operators are closed and from equation \eqref{Eqn: boundary terms} it follows that their adjoints are the following:
	\begin{align*}
		\operatorname{dom}(\mathrm{d}_\Sigma)&:=\lbrace
		\omega\in\mathrm{L}^2\Omega^k(\Sigma)|\;\mathrm{d}_\Sigma\omega\in\mathrm{L}^2\Omega^{k+1}(\Sigma)\rbrace\,,\qquad
		\delta_{\Sigma,\mathrm{n}}^*=\mathrm{d}_\Sigma\,,\\
		\operatorname{dom}(\delta_\Sigma)&:=\lbrace
		\omega\in\mathrm{L}^2\Omega^k(\Sigma)|\;\delta_\Sigma\omega\in\mathrm{L}^2\Omega^{k-1}(\Sigma)\rbrace\,,\qquad
		\mathrm{d}_{\Sigma,\mathrm{t}}^*=\delta_\Sigma\,.
	\end{align*}
	It then follows immediately that $(\overline{\operatorname{Ran}(\mathrm{d}_{\Sigma,\mathrm{t}})}\oplus\overline{\operatorname{Ran}(\delta_{\Sigma,\mathrm{n}})})^\perp=\ker(\mathrm{d})\cap\ker\delta=\mathcal{H}^k(\Sigma)$ so that
	\begin{align}\label{Eqn: weak-Hodge decomposition for boundary}
		\mathrm{L}^2\Omega^k(\Sigma)=
		\overline{\operatorname{Ran}(\mathrm{d}_{\Sigma,\mathrm{t}})}\oplus
		\overline{\operatorname{Ran}(\delta_{\Sigma,\mathrm{n}})}\oplus
		\mathcal{H}^k(\Sigma)\,.
	\end{align}
	Following the same steps of proof of theorem \ref{Thm: Hodge decomposition for manifolds with interface} it follows that a similar weak-Hodge decomposition holds for the case of non-compact Riemannian manifolds $\Sigma$ with interface $Z$, actually
	\begin{align}\label{Eqn: weak-Hodge decomposition for interface}
		\mathrm{L}^2\Omega^k(\Sigma)=
		\overline{\operatorname{Ran}(\mathrm{d}_{\Sigma,[\mathrm{t}]})}\oplus
		\overline{\operatorname{Ran}(\delta_{\Sigma,[\mathrm{n}]})}\oplus
		\mathcal{H}^k(\Sigma)\,,
	\end{align}
	where $\mathrm{d}_{\Sigma,[\mathrm{t}]}, \delta_{\Sigma,[\mathrm{n}]}$ are defined by
	\begin{align*}
	\operatorname{dom}(\mathrm{d}_{\Sigma,[\mathrm{t}]})&:=\lbrace
	\omega\in\mathrm{L}^2\Omega^k(\Sigma)|\;\mathrm{d}_\Sigma\omega\in\mathrm{L}^2\Omega^{k+1}(\Sigma)\,,\;[\mathrm{t}\omega]=0\rbrace\qquad
	\mathrm{d}_{\Sigma,[\mathrm{t}]}\omega:=\mathrm{d}_\Sigma\omega\,,\\
	\operatorname{dom}(\delta_{\Sigma,[\mathrm{n}]})&:=\lbrace
	\omega\in\mathrm{L}^2\Omega^k(\Sigma)|\;\delta_\Sigma\omega\in\mathrm{L}^2\Omega^{k-1}(\Sigma)\,,\;[\mathrm{n}\omega]=0\rbrace\qquad
	\delta_{\Sigma,[\mathrm{n}]}\omega:=\delta_\Sigma\omega\,.
	\end{align*}
	This time $\mathrm{d}_{\Sigma,[\mathrm{t}]}^*=\delta_{\Sigma,[\mathrm{n}]}$ as well as $\delta_{\Sigma,[\mathrm{n}]}^*=\mathrm{d}_{\Sigma,[\mathrm{t}]}$ so that in particular $\ker\mathrm{d}_{\Sigma,[\mathrm{t}]}^*\cap\ker\delta_{\Sigma,[\mathrm{n}]}=\mathcal{H}^k(\Sigma)$.
	\\\nicomment{(The notation is sloppy, in principle $\mathrm{d}_{\Sigma,\mathrm{t}},\delta_{\Sigma,\mathrm{n}}$ depend on $k$.)}
\end{remark}
\begin{remark}[Non-dynamical Maxwell equations]\label{Rmk: Hodge formulation of non-dynamical Maxwell equations}
	The Hodge decomposition with interface proved in theorem \ref{Thm: Hodge decomposition for manifolds with interface} can be exploited to formulate the correct generalization of the non-dynamical equations \eqref{Eqn: non-dynamical Maxwell equations}.
	Actually in what follows we will substitute equations \eqref{Eqn: non-dynamical Maxwell equations} with the requirement
	\begin{align}\label{Eqn: non-dynamical Maxwell equations with Hodge decomposition}
		E,B\perp\mathrm{d}_\Sigma\mathrm{H}^1\Omega^0_{[\mathrm{t}]}(\Sigma_Z)\,.
	\end{align}
	Notice that this entails $\delta_\Sigma E=\delta_\Sigma B=0$ as well as $[\mathrm{n}E]=[\mathrm{n}B]=0$.
	Configurations of the electric field $E$ in the presence of a charge density $\rho$ on $\Sigma_\pm$ and a surface charge density $\rho_Z$ over $Z$ are described by expanding $E=\mathrm{d}_\Sigma\alpha+\delta_\Sigma\beta+\kappa$ and demanding $\alpha\in\mathrm{H}^1\Omega^0_{[\mathrm{t}]}(\Sigma_Z)$ to satisfy
	\begin{align*}
		(\mathrm{d}_\Sigma\varphi,\mathrm{d}_\Sigma\alpha)_\Sigma=
		(\varphi,\rho)_\Sigma+(\varphi,\rho)_{Z}\qquad
		\forall\varphi\in C^\infty_{\mathrm{c}}(\Sigma)\,.
	\end{align*}
	For sufficiently regular $\alpha$ this is equivalent to the Poisson problem $\Delta_\Sigma\alpha=\rho$, $[\mathrm{n}\mathrm{d}_\Sigma\alpha]=\rho_Z$, recovering the classical equations outlined in \cite[Sec. I.5]{Jackson-99}.
\end{remark}

\subsection{Dynamical equations: lagrangian subspaces}\label{Sec: dynamical equations: boundary triples}
In this section we will deal with the dynamical equations \eqref{Eqn: dynamical Maxwell equations}.
These can be easily regarded as a Schr\"odinger equation and solved by imposing suitable boundary conditions on $Z$.
Actually we can rewrite equations \eqref{Eqn: dynamical Maxwell equations} in Schr\"odinger form
\begin{align}\label{Eqn: dynamical eqns in Schroedinger form}
	i\partial_t\psi=H\psi\,\qquad
	\psi:=\bigg[\begin{matrix}E\\B\end{matrix}\bigg]\,,\qquad
	H:=\bigg[\begin{matrix}0&i\operatorname{curl}\\-i\operatorname{curl}&0\end{matrix}\bigg]\,,
\end{align}
Here we adopt the convention of \cite{Baer-19} according to which
\begin{align}\label{Eqn: curl convention}
	\operatorname{curl}:=i\ast_\Sigma\mathrm{d}_\Sigma\quad\textrm{if }\dim\Sigma=1\mod 4\,,\qquad
	\operatorname{curl}:=\ast_\Sigma\mathrm{d}_\Sigma\quad\textrm{if }\dim\Sigma=3\mod 4\,.
\end{align}
Which this convention $\operatorname{curl}$ is formally selfadjoint on $C^\infty_{\mathrm{c}}H^1\Omega^1(\Sigma)$.

As in section \ref{Sec: Non-dynamical equations: Hodge theory with interface} we wish to consider equation \eqref{Eqn: dynamical eqns in Schroedinger form} on $\Sigma_Z$, allowing for jump discontinuities across the interface $Z$.
For that we regard $H$ as a densely defined operator on $\mathrm{L}^2\Omega^1(\Sigma)^{\times 2}=\mathrm{L}^2\Omega^1(\Sigma_Z)^{\times 2}$ with domain
\begin{align}\label{Eqn: curl-Hamiltonian domain}
	\operatorname{dom}(H):=
	C^\infty_{\mathrm{cc}}\Omega^1(\Sigma_+)\oplus C^\infty_{\mathrm{cc}}\Omega^1(\Sigma_-)\,,
\end{align}
where $C^\infty_{\mathrm{cc}}\Omega^1(\Sigma_\pm)$ denotes the subspace of $C^\infty_{\mathrm{c}}\Omega^1(\Sigma_\pm)$ with support in $\Sigma_\pm\setminus\partial\Sigma_\pm$.
The operator $H$ is closable and symmetric, its adjoint $H^*$ being defined by
\begin{align}\label{Eqn: adjoint curl-Hamiltonian}
	\operatorname{dom}(H^*)=\lbrace
	\psi\in\mathrm{L}^2\Omega^1(\Sigma)^{\times 2}|\;H\psi\in\mathrm{L}^2\Omega^1(\Sigma)^{\times 2}\rbrace\qquad
	H^*\psi:=H\psi\,.
\end{align}
Equation \eqref{Eqn: dynamical eqns in Schroedinger form} is solved by selecting a self-adjoint extension of $H$.
The latter can be parametrized by Lagrangian subspaces of a suitable complex symplectic space -- \cite{Everitt-Markus-99,Everitt-Markus-03,Everitt-Markus-05}.
\begin{Definition}\label{Def: complex symplectic space, Lagrangian subspaces}
	Let $\mathsf{S}$ be a complex vector space and let $\sigma\colon\mathsf{S}\times\mathsf{S}\to\mathbb{C}$ a bilinear map.
	The pair $(\mathsf{S},\sigma)$ is called complex symplectic space if $\sigma$ is non-degenerate -- \textit{i.e.} $\sigma(x,y)=0$ for all $y\in\mathsf{S}$ implies $x=0$ -- and $\sigma(x,y)=-\overline{\sigma(y,x)}$ for all $x,y\in\mathsf{S}$.
	A subspace $L\subseteq\mathsf{S}$ is called Lagrangian subspace if $L=L^\perp:=\lbrace x\in\mathsf{S}|\;\sigma(x,y)=0\;\forall y\in L\rbrace$.
\end{Definition}
For convenience we summarize the major results in the following theorem:
\begin{theorem}[\cite{Everitt-Markus-99}]\label{Thm: self-adjoint extensions with Lagrangian subspaces}
	Let $\mathsf{H}$ a separable Hilbert space and let $A\colon\operatorname{dom}(A)\subseteq\mathsf{H}\to\mathsf{H}$ be a symmetric operator.
	Then the bilinear map
	\begin{align}\label{Eqn: symplectic form}
		\sigma(x,y):=(A^*x,y)-(x,A^*y)\,,\qquad\forall x,y\in\operatorname{dom}(A^*)\,,
	\end{align}
	satisfies $\sigma(x,y)=-\overline{\sigma(y,x)}$.
	It also descends to the quotient space $\mathsf{S}_A:=\operatorname{dom}(A^*)/\operatorname{dom}(A)$ and the pair $(\mathsf{S}_A,\sigma)$ is a complex symplectic space as per definition \ref{Def: complex symplectic space, Lagrangian subspaces}.
	Moreover, for all Lagrangian subspace $L\subseteq\mathsf{S}_A$  -- \textit{cf.} definition \ref{Def: complex symplectic space, Lagrangian subspaces} -- the operator
	\begin{align}\label{Eqn: Lagrangian self-adjoint extension}
		A_L:=A^*|_{L+\operatorname{dom}(A)}\,,
	\end{align}
	defines a self-adjoint extension of $A$ -- here $L+\operatorname{dom}(A)$ denotes the pre-image of $L$ with respect to the projection $\operatorname{dom}(A^*)\to\mathsf{S}_A$.
	Finally the map
	\begin{align}
		\lbrace\textrm{Lagrangian subspaces $L$ of $\mathsf{S}_A$}\rbrace
		\ni L\mapsto A_L\in
		\lbrace\textrm{self-adjoint extensions of }A\rbrace\,,
	\end{align}
	is one-to-one.
\end{theorem}
\begin{Example}
	As a concrete example of theorem \ref{Thm: self-adjoint extensions with Lagrangian subspaces} we discuss the case of the self-adjoint extensions of the $\operatorname{curl}$ operator on a closed manifold $\Sigma$ with interface $Z$.
	For simplicity we assume that $\dim\Sigma=2k+1$ with $\dim\Sigma=3\mod 4$, while $\operatorname{curl}$ is defined according to \eqref{Eqn: curl convention}.
	We consider the operator $\operatorname{curl}_Z$ defined by
	\begin{align}\label{Eqn: Z-curl operator}
		\operatorname{dom}(\operatorname{curl}_Z):=\overline{C^\infty_{\mathrm{c}}\Omega^k(\Sigma_Z)}^{\|\|_{\operatorname{curl}}}\,,\qquad
		\operatorname{curl}_Zu:=\operatorname{curl}u\,.
	\end{align}
	Notice that $C^\infty_{\mathrm{c}}\Omega^k(\Sigma_Z)=C^\infty_{\mathrm{cc}}\Omega^k(\Sigma_+)\oplus C^\infty_{\mathrm{cc}}\Omega^k(\Sigma_-)$.
	The adjoint $\operatorname{curl}_Z^*$ of $\operatorname{curl}_Z$ is
	\begin{align}\label{Eqn: adjoint of Z-curl operator}
		\operatorname{dom}(\operatorname{curl}_Z^*)&=
		\operatorname{dom}(\operatorname{curl}_+)\oplus\operatorname{dom}(\operatorname{curl}_-)\,,\\
		\operatorname{dom}(\operatorname{curl}_\pm)&:=\lbrace
		u_\pm\in\mathrm{L}^2\Omega^k(\Sigma_\pm)|\;\operatorname{curl}_\pm u_\pm\in\mathrm{L}^2\Omega^k(\Sigma_\pm)\rbrace\,,\qquad
		\operatorname{curl}_Zu:=\operatorname{curl}u\,.
	\end{align}
	Since complex conjugation commutes with $\operatorname{curl}$, it follows from Von Neumann's criterion \cite[Thm. 5.43]{Moretti-18} that $\operatorname{curl}_Z$ admits self-adjoints extensions.
	We now provide a description of the complex symplectic space $\mathsf{S}_{\operatorname{curl}_Z}:=(\operatorname{dom}(\operatorname{curl}_Z^*)/\operatorname{dom}(\operatorname{curl}_Z),\sigma_Z)$ whose Lagrangian subspaces allows to characterize all self-adjoint extensions of $\operatorname{curl}_Z$.
	According to theorem \ref{Thm: self-adjoint extensions with Lagrangian subspaces} the symplectic structure $\sigma_Z$ on the vector space $\mathsf{S}_{\operatorname{curl}_Z}$ is defined by
	\begin{align}\label{Eqn: presymplectic structure over the adjoint of Z-curl operator}
		\sigma_Z(u,v):=
		(\operatorname{curl}_Z^*u,v)-(u,\operatorname{curl}_Z^*v)\,,\qquad
		\forall u,v\in\operatorname{dom}(\operatorname{curl}_Z^*)\,.
	\end{align}
	In particular for $u\in\operatorname{dom}(\operatorname{curl}_Z^*)$ and $v\in\mathrm{H}^1\Omega^k(\Sigma_Z)$ we have
	\begin{align}\label{Eqn: simplified form for presymplectic structure}
		\sigma(u,v)&=
		\sum_\pm\pm\int_Z\overline{\mathrm{t}_\pm u}\wedge\mathrm{t}_\pm v=
		\sum_\pm\mp\prescript{}{-\frac{1}{2}}{\langle}\mathrm{t}_\mp u,\ast_Z\mathrm{t}_\mp v\rangle_{\frac{1}{2}}\\&=
		(\gamma_1u,\gamma_0v)-(\gamma_0u,\gamma_1v)\,,\qquad
		\gamma_0u:=\frac{1}{\sqrt{2}}\ast_Z[\mathrm{t}u]\,,\,\qquad
		\gamma_1u:=\frac{1}{\sqrt{2}}(\mathrm{t}_+u+\mathrm{t}_-u)\,.\,,
	\end{align}
	where $\prescript{}{-\frac{1}{2}}{\langle}\;,\;\rangle_{\frac{1}{2}}$ denotes the pairing between $\mathrm{H}^{-\frac{1}{2}}\Omega^k(Z)$ and $\mathrm{H}^{\frac{1}{2}}\Omega^k(Z)$.
	In particular this shows that $\mathrm{t}_\pm u\in\mathrm{H}^{-\frac{1}{2}}\Omega^k(Z)$ for all $u\in\operatorname{dom}(\operatorname{curl}_Z^*)$ -- \textit{cf.} \cite{Alonso-Valli-96,Buffa-Costabel-Sheen-02,Georgescu-79,Paquet-82,Weck-04} for more details on the trace space associated with the $\operatorname{curl}$-operator on a manifold with boundary.
	\nicomment{Provide more details.}
	
	According to theorem \ref{Thm: self-adjoint extensions with Lagrangian subspaces} all self-adjoint extensions of $\operatorname{curl}_Z$ are in one-to-one correspondence to the Lagrangian subspaces of $\mathsf{S}_{\operatorname{curl}_Z}$.
	Unfortunately a complete characterization of all Lagrangian subspaces of $\mathsf{S}_{\operatorname{curl}_Z}$ is not at disposal.
	We content ourself to present a family of Lagrangian subspaces -- a generalization of the results presented in \cite{Hiptmair-Kotiuga-Tordeux-12} may provide other examples.
	For $\theta\in\mathbb{R}$ let
	\begin{align}
		L_{\theta}:=\lbrace
		u\in\operatorname{dom}(\operatorname{curl}_Z^*)|\;\mathrm{t}_+u=e^{i\theta}\mathrm{t}_-u\rbrace\,,
	\end{align}
	where $[\mathrm{t}u]=\mathrm{t}_+u-\mathrm{t}_-u$ denotes the tangential jump -- \textit{cf.} definition \ref{Def: jump of tangential and normal component}, remark \ref{Rmk: extension of tangential and normal maps to Sobolev spaces} and equation \eqref{Eqn: simplified form for presymplectic structure}.
	To show that $L_\theta$ are Lagrangian subspaces let $u,v\in L_\theta$ and let $v_n\in\mathrm{H}^1\Omega^k(\Sigma_Z)$ be such that $\|v-v_n\|_{\operatorname{curl}}\to 0$.
	In particular $\|(\mathrm{t}_+-e^{i\theta}\mathrm{t}_-) v\|_{\mathrm{H}^{\frac{1}{2}}\Omega^k(\Sigma_Z)}\to 0$ so that
	\begin{align}
		\sigma_Z(u,v)=
		\lim_n\sigma_Z(u,v_n)=
		-\lim_n
		\prescript{}{-\frac{1}{2}}{\langle}\mathrm{t}_+u,\ast_Z(\mathrm{t}_+v_n-e^{i\theta}\mathrm{t}_-v_n)\rangle_{\frac{1}{2}}=0\,.
	\end{align}
	It follows that $L_\theta\subseteq L_\theta^\perp$.
	Conversely if $u\in L_\theta^\perp$ let consider $v\in C^\infty_{\mathrm{c}}\Omega^k(\Sigma_Z)\cap L_\theta$.
	Since $u\in L_\theta^\perp$ we find
	\begin{align*}
		0=\sigma_Z(u,v)=
		-\prescript{}{-\frac{1}{2}}{\langle}\mathrm{t}_+u-e^{i\theta}\mathrm{t}_-u,\ast_Z\mathrm{t}_+v\rangle_{\frac{1}{2}}\,.	
	\end{align*}
	Since $\mathrm{t}_+\colon C^\infty_{\mathrm{c}}\Omega^k(\Sigma_+)\to C^\infty_{\mathrm{c}}\Omega^k(Z)$ is surjective it follows that $\mathrm{t}_+u=e^{i\theta}\mathrm{t}_-u$.

	Notice that the self-adjoint extension obtained for $\theta=0$ coincides with the closure of $\operatorname{curl}$ on $C^\infty_{\mathrm{c}}(\Sigma)$ which is known to be self-adjoint by \cite[Lem. 2.6]{Baer-19}.
	\\\nicomment{(Here we exploited the compactness property.
		We may deal with non-compact manifolds too because \cite[Lem. 2.6]{Baer-19} is based on point \textit{(i)} of \cite[Lem. 2.3]{Baer-19} which holds true also in this setting.)}
	Indeed, since $[\mathrm{t}]$ is continuous we have $\operatorname{dom}(\overline{\operatorname{curl}})\subseteq L_{0}$ so that $\operatorname{curl}_{Z,L_0}$ is a self-adjoint extension of $\overline{\operatorname{curl}}$.
	Since the latter operator is already self-adjoint we have equality among the two.
\end{Example}


We conclude this section by introducing an exact sequence which provides a complete description of the solution space of the Maxwell equations \eqref{Eqn: dynamical Maxwell equations} with interface $Z$.
For that the following standard definition is in order \nicomment{[Cit.]}
\begin{Definition}\label{Def: future,past,timelike-compact functions}
	In the hypothesis of theorem \ref{Thm: boundary triple} let $\Theta\colon\mathsf{h}\to\mathsf{h}$ be a self-adjoint operator and consider the self-adjoint extension $H_\Theta$ as per theorem \ref{Thm: boundary triple} and let $\mathrm{H}^\infty_\Theta\Omega^1(\Sigma_Z):=\bigcap_{\ell\geq 1}\mathrm{dom}(H_\Theta^\ell)$.
	We define the following subspaces of $C^\infty(\mathbb{R},\mathrm{H}^\infty_\Theta\Omega^1(\Sigma_Z))$:
	\begin{itemize}
		\item
		the space $C^\infty_{\mathrm{sfc}}(\mathbb{R},\mathrm{H}^\infty_\Theta\Omega^1(\Sigma_Z))$ of future-compact functions, made of those $f\in C^\infty(\mathbb{R},\mathrm{H}^\infty_\Theta\Omega^1(\Sigma_Z))$ such that $J^-(x)\cap\operatorname{spt}(f)$ is compact for all $x\in M$.
		Here $J^-(x)$ denotes the causal past of $x\in M=\mathbb{R}\times\Sigma$ according to the causal structure induced by $g=-\mathrm{d}t^2+h$;
		\item
		the space $C^\infty_{\mathrm{spc}}(\mathbb{R},\mathrm{H}^\infty_\Theta\Omega^1(\Sigma_Z))$ of past-compact functions, made of those $f\in C^\infty(\mathbb{R},\mathrm{H}^\infty_\Theta\Omega^1(\Sigma_Z))$ such that $J^+(x)\cap\operatorname{spt}(f)$ is compact for all $x\in M$.
		Here $J^+(x)$ denotes the causal future of $x\in M=\mathbb{R}\times\Sigma$ according to the causal structure induced by $g=-\mathrm{d}t^2+h$;
		\item 
		the space of timelike-compact functions defined by $$C^\infty_{\mathrm{tc}}(\mathbb{R},\mathrm{H}^\infty_\Theta\Omega^1(\Sigma_Z))=C^\infty_{\mathrm{sfc}}(\mathbb{R},\mathrm{H}^\infty_\Theta\Omega^1(\Sigma_Z))\cap C^\infty_{\mathrm{spc}}(\mathbb{R},\mathrm{H}^\infty_\Theta\Omega^1(\Sigma_Z))\,.$$
\end{itemize}
\end{Definition}
\nicomment{(Perhaps it is worth to recall somewhere the notion of advanced-retarded fundamental solutions?)}
\begin{proposition}\label{Prop: exact sequence for Maxwell equations with interface}
	Let $(\Sigma,h)$ be a closed manifold with interface $Z\stackrel{\iota_Z}{\hookrightarrow}\Sigma$.
	Moreover let $(\Sigma_\pm,h_\pm)$ the oriented, compact Riemannian manifolds with boundary $\partial\Sigma_\pm=\pm Z$ such that $\Sigma_Z:=
	\Sigma\setminus Z=\Sigma_+\cup\Sigma_-$.
	Let $H$ be the densely defined operator on $\mathrm{L}^2\Omega^1(\Sigma)$ with domain defined by \eqref{Eqn: curl-Hamiltonian domain} and let $H^*$ be its adjoint, defined as in \eqref{Eqn: adjoint curl-Hamiltonian}.
	Let $(\mathsf{h},\gamma_0,\gamma_1)$ be the boundary triple associated with $H^*$ as described in proposition \ref{Proposition: boundary triple for curl-Hamiltonian} and let $\Theta\colon\mathsf{h}\to\mathsf{h}$ be a self-adjoint operator and consider the self-adjoint extension $H_\Theta$ as per theorem \ref{Thm: boundary triple}.
	Let $G^\pm_\Theta$ be the operators $G^\pm_\Theta\colon C^\infty_{\mathrm{tc}}(\mathbb{R},\mathrm{H}^\infty_\Theta\Omega^1(\Sigma_Z))\to C^\infty(\mathbb{R},\mathrm{H}^\infty_\Theta\Omega^1(\Sigma_Z))$ defined by
	\begin{align}
		(G^\pm_\Theta \omega)(t)=\int_{\mathbb{R}}\theta(\pm(t-s))e^{-i(t-s)H_\Theta}\omega(s)\mathrm{d}s\,.
	\end{align}
	The the operator $G^+_\Theta$ (\textit{resp}. $G^-_\Theta$) is an advanced (\textit{resp}. retarded) solution of $i\partial_t+H_\Theta$, that is, it holds
	\begin{align}
		(i\partial_t+H_\Theta)\circ G_\Theta^\pm|_{C^\infty_{\mathrm{tc}}(\mathbb{R},\mathrm{H}^\infty_\Theta\Omega^1(\Sigma_Z))}=
		\operatorname{Id}_{C^\infty_{\mathrm{tc}}(\mathbb{R},\mathrm{H}^\infty_\Theta\Omega^1(\Sigma_Z))}\,,\\
		G_\Theta^\pm\circ(i\partial_t+H_\Theta)|_{C^\infty_{\mathrm{tc}}(\mathbb{R},\mathrm{H}^\infty_\Theta\Omega^1(\Sigma_Z))}=
		\operatorname{Id}_{C^\infty_{\mathrm{tc}}(\mathbb{R},\mathrm{H}^\infty_\Theta\Omega^1(\Sigma_Z))}\,.
	\end{align}
	Moreover, let $G_\Theta:=G^+_\Theta-G^-_\Theta$.
	Then the following is a short exact sequence
	\begin{align}
	\nonumber
	0\to
	C^\infty_{\mathrm{tc}}(\mathbb{R},\mathrm{H}^\infty_\Theta\Omega^1(\Sigma_Z))
	&\stackrel{i\partial_t+H_\Theta}{\to}
	C^\infty_{\mathrm{tc}}(\mathbb{R},\mathrm{H}^\infty_\Theta\Omega^1(\Sigma_Z))\\&
	\label{Eqn: exact sequence}
	\stackrel{G_\Theta}{\to}
	C^\infty(\mathbb{R},\mathrm{H}^\infty_\Theta\Omega^1(\Sigma_Z))
	\stackrel{i\partial_t+H_\Theta}{\to}
	C^\infty(\mathbb{R},\mathrm{H}^\infty_\Theta\Omega^1(\Sigma_Z))
	\to 0\,.
	\end{align}
\end{proposition}
\begin{proof}
	Most of it is an analogue of \cite[Thm. 30- Prop. 36]{Dappiaggi-Drago-Ferreira-19}.
	The finite speed of propagation follows from \cite{Higson-Roe-00,Mcintosh-Morris-13}.
\end{proof}
\begin{remark}
	Notice that the exact sequence \eqref{Eqn: exact sequence} implies that the space of smooth solution of the dynamical equations \eqref{Eqn: dynamical Maxwell equations} is isomorphic as a vector space to the image of $G_\Theta$.
\end{remark}


%\section{Boundary triples}
%
%
%boundary triples, which we recall briefly in the following -- \textit{cf.} \cite{Behrndt-Langer-12} and references there in.
%\begin{Definition}\label{Def: boundary triple}
%	Let $\mathsf{H}$ a separable Hibert space and let $A\colon\operatorname{dom}(A)\subseteq\mathsf{H}\to\mathsf{H}$ be a symmetric operator.
%	A boundary triple for $A^*$ is a triple $(\mathsf{h},\gamma_0,\gamma_1)$ made of an Hilbert space $\mathsf{h}$ and a pair of linear maps $\gamma_0,\gamma_1$ such that $\gamma_0\oplus\gamma_1\colon\operatorname{dom}(A^*)\to\mathsf{h}\oplus\mathsf{h}$ is continuous and surjective and moreover
%	\begin{align}\label{Eqn: boundary triple relation}
%	(A^*x,y)_{\mathsf{H}}-(x,A^*y)_{\mathsf{H}}=(\gamma_1x,\gamma_0y)_{\mathsf{h}}-(\gamma_0x,\gamma_1y)_{\mathsf{h}}\qquad\forall x,y\in\operatorname{dom}(A^*)\,,
%	\end{align}
%	where $(\;,\;)_{\mathsf{H}}$, $(\;,\;)_{\mathsf{h}}$ denotes the scalar product on $\mathsf{H}$ and $\mathsf{h}$ respectively.
%\end{Definition}
%We refer to \cite{Behrndt-Langer-12} for a extensive treatment of boundary triples.
%For the application we have in mind we only need the following properties of boundary triples which we recollect in the following
%\begin{theorem}\label{Thm: boundary triple}
%	Let $\mathsf{H}$ a separable Hibert space and let $A\colon\operatorname{dom}(A)\subseteq\mathsf{H}\to\mathsf{H}$ be a symmetric operator.
%	Let $(\mathsf{h},\gamma_0,\gamma_1)$ be a boundary triple for $A^*$ as per definition \ref{Def: boundary triple} and let $\Theta$ be a closed relation on $\mathsf{h}\times\mathsf{h}$.
%	\begin{enumerate}
%		\item 
%		the operator $A_\Theta\colon\operatorname{dom}(A_\Theta)\subseteq\mathsf{H}\to\mathsf{H}$ defined by
%		\begin{align}
%		\operatorname{dom}(A_\Theta):=\ker(\gamma_1-\Theta\gamma_1)\qquad
%		A_\Theta:=A^*|_{\operatorname{dom}(A_\Theta)}\,,
%		\end{align}
%		is a closed extension of $A$;
%		\item 
%		$A_\Theta^*=A_{\Theta^*}$ ; in particular $A_\Theta$ is selfadjoint if and only if $\Theta$ is selfadjoint ;
%		\item
%		the assignment $\Theta\mapsto A_\Theta$ leads to a bijective map
%		\begin{align}
%		\lbrace\textrm{closed relations }\Theta\rbrace
%		\ni\Theta\mapsto A_\Theta\in
%		\lbrace\textrm{closed extensions of }A\rbrace\,.
%		\end{align}
%	\end{enumerate}
%\end{theorem}
%We now present a boundary triple for the operator $H$ defined in \eqref{Eqn: curl-Hamiltonian domain}.
%\begin{remark}\label{Rmk: trace space for curl-Hamiltonian}
%	In the following we shall exploit a few basic facts which we shall recall here for convenience.
%	First of all the map $[\mathrm{t}]\colon\mathrm{H}^{\ell}\Omega^1(\Sigma_Z)\to\mathrm{H}^{\ell-\frac{1}{2}}\Omega^1(Z)$ -- \textit{cf.} remark \ref{Rmk: extension of tangential and normal maps to Sobolev spaces} -- extends to a continuous surjection
%	\begin{align}
%	[\mathrm{t}]\colon\operatorname{dom}(H^*)\to\Omega^1_{\mathrm{Tr}}(Z):=
%	\lbrace\omega\in\mathrm{H}^{-\frac{1}{2}}\Omega^1(Z)|\;\delta_Z\omega\in \mathrm{H}^{-\frac{1}{2}}\Omega^0(Z)\rbrace\,.
%	\end{align}
%	Details of this construction can be found in \cite{Alonso-Valli-96,Paquet-82,Georgescu-79,Weck-04} for the case of a Riemannian manifold with boundary.
%	The space $\Omega_{\mathrm{Tr}}^1(Z)$ is an Hilbert space with scalar product
%	\begin{align*}
%	(\omega_1,\omega_2)_{\Omega_{\mathrm{Tr}}^1(Z)}:=
%	(\omega_1,\omega_2)_{\mathrm{H}^{-\frac{1}{2}}\Omega^1(Z)}+
%	(\delta_Z\omega_1,\delta_Z\omega_2)_{\mathrm{H}^{-\frac{1}{2}}\Omega^1(Z)}\,.
%	\end{align*}
%	For later convenience we introduce a pair of isomorphisms $\iota_\pm\colon\mathrm{H}^{\pm\frac{1}{2}}\Omega^1(\Sigma_Z)\to\Omega_{\mathrm{Tr}}^1(Z)$ defined by
%	\begin{align}\label{Eqn: isomorphism from half-Sobolev space to trace space for curl-Hamiltonian}
%	(\iota_+\omega_+,\iota_-\omega_-)_{\Omega_{\mathrm{Tr}}^1(Z)}=
%	\;_{\frac{1}{2}}\langle\omega_+,\omega_-\rangle_{-\frac{1}{2}}
%	\qquad\forall\omega_\pm\in\mathrm{H}^{\pm\frac{1}{2}}\Omega^1(\Sigma_Z)\,,
%	\end{align}
%	where $_{\frac{1}{2}}\langle\;,\;\rangle_{-\frac{1}{2}}$ denotes the dual pairing between $\mathrm{H}^{\frac{1}{2}}\Omega^1(\Sigma_Z)$ and $\mathrm{H}^{-\frac{1}{2}}\Omega^1(\Sigma_Z)$ -- \textit{cf.} \cite{Behrndt-Langer-12}.
%\end{remark}
%\begin{remark}\label{Rmk: Dirichlet-curl Hamiltonian extension}
%	Let consider the self-adjoint extension $H_\infty:=H^*|_{\operatorname{dom}(H_\infty)}$ where
%	\begin{align}\label{Eqn: Dirichlet-curl Hamiltonian extension}
%	\operatorname{dom}(H_\infty):=\lbrace
%	\psi\in\mathrm{L}^2\Omega^1(\Sigma)^{\times 2}|\;H\psi\in\mathrm{L}^2\Omega^1(\Sigma)^{\times 2}
%	\rbrace=
%	\mathrm{H}^1\Omega^1(\Sigma)^{\times 2}\,,
%	\end{align}
%	where in the second equality we exploited \cite[Thm. 3.1.1]{Schwarz-95}.
%	The self-adjoint extension $H_\infty$ coincides with the unique self-adjoint extension of the operator $H$ defined on the domain $C^\infty_{\mathrm{c}}\Omega^1(\Sigma)$ and corresponds to the standard Hamiltonian operator for the system \eqref{Eqn: dynamical eqns in Schroedinger form} in the case of empty interface $Z$.
%	The spectrum of the latter operator on closed manifolds is known \cite{Baer-19} in particular $\pm\lambda\in\sigma(H_\infty)$ if and only if $\lambda\in\sigma(\operatorname{curl})$ where $\operatorname{curl}$ is the (essentially selfadjoint) curl operator \eqref{Eqn: curl convention} defined on $C^\infty_{\mathrm{c}}\Omega^k(\Sigma)$.
%	In particular $\sigma(H_\infty)$ has no continuous spectrum and its point spectrum consists of the eigenvalue $0$ (with infinite multiplicity) and the discrete spectrum (with finite multiplicity) -- \textit{cf.} \cite[Thm. 3.1]{Baer-19}. 
%	Moreover, on account of \cite[Lem. 21]{Dappiaggi-Drago-Ferreira-19} -- see also \cite{Behrndt-Langer-12} -- we have that for all $\lambda\in\mathbb{R}\cap\rho(H_\infty)$ it holds
%	\begin{align}\label{Eqn: decomposition of domain of curl-Hamiltonian}
%	\mathrm{dom}(H^*)=\mathrm{dom}(H_\infty)\dotplus\ker(H^*-\lambda)\,,
%	\end{align}
%	where $\cdot$ denotes algebraic sum.
%	\\\nicomment{(Decomposition \eqref{Eqn: decomposition of domain of curl-Hamiltonian} needs some information about $\sigma(H_\infty)$ -- e.g. the fact that $\sigma(H_\infty)\neq\mathbb{R}$ -- which are proved in \cite{Baer-19} for compact manifolds.)}
%\end{remark}
%\begin{proposition}\label{Proposition: boundary triple for curl-Hamiltonian}
%	Let $(\Sigma,h)$ be a closed manifold with interface $Z\stackrel{\iota_Z}{\hookrightarrow}\Sigma$.
%	Moreover let $(\Sigma_\pm,h_\pm)$ the oriented, compact Riemannian manifolds with boundary $\partial\Sigma_\pm=\pm Z$ such that $\Sigma_Z:=
%	\Sigma\setminus Z=\Sigma_+\cup\Sigma_-$.
%	Let $H$ be the densely defined operator on $\mathrm{L}^2\Omega^1(\Sigma)$ with domain defined by \eqref{Eqn: curl-Hamiltonian domain} and let $H^*$ be its adjoint, defined as in \eqref{Eqn: adjoint curl-Hamiltonian}.
%	Finally let $H_\infty$ be the self-adjoint extension defined in \eqref{Eqn: Dirichlet-curl Hamiltonian extension} and consider $\lambda\in\mathbb{R}\cap\rho(H_\infty)$.
%	Then the following is a boundary triple for the adjoint operator $H^*$ defined as in \eqref{Eqn: adjoint curl-Hamiltonian}:
%	\begin{align}\label{Eqn: boundary triple for curl-Hamiltonian}
%	\mathsf{h}:=\Omega_{\mathrm{Tr}}^1(Z)^{\times 2}\,\qquad
%	\gamma_0
%	\bigg[\begin{matrix}E\\B\end{matrix}\bigg]:=
%	\frac{i}{\sqrt{2}}
%	\bigg[\begin{matrix}
%	\iota_-\ast_\Sigma[\mathrm{t}B]\\
%	\iota_-\ast_\Sigma[\mathrm{t}E]
%	\end{matrix}\bigg]\,,\qquad
%	\gamma_1\bigg[\begin{matrix}E\\B\end{matrix}\bigg]:=
%	\frac{i}{\sqrt{2}}
%	\bigg[\begin{matrix}
%	\iota_+\ast_\Sigma(\mathrm{t}_-+\mathrm{t}_+)B_\infty\\
%	\iota_+\ast_\Sigma(\mathrm{t}_-+\mathrm{t}_+)E_\infty
%	\end{matrix}\bigg]\,.
%	\end{align}
%	Here $E_\infty,B_\infty\in\operatorname{dom}(H_\infty)$ denote the components of $E,B$ according to the decomposition \eqref{Eqn: decomposition of domain of curl-Hamiltonian} -- \textit{cf.} remark \ref{Rmk: Dirichlet-curl Hamiltonian extension} -- while $\iota_\pm$ have been introduced in \eqref{Eqn: isomorphism from half-Sobolev space to trace space for curl-Hamiltonian}.
%	Finally $\mathrm{t}_\pm$ denote the tangential traces $\Sigma_\pm$ as per definitions \ref{Def: tangential and normal component} while $[\mathrm{t}E]$ is the tangential jump as per definition \ref{Def: jump of tangential and normal component}.
%\end{proposition}
%\begin{proof}
%	The proof follows \cite[Prop. 6.9]{Behrndt-Langer-12}.
%	First of all notice that $\gamma_0,\gamma_1$ are well-defined, continuous and surjective -- \textit{cf.} remark \ref{Rmk: trace space for curl-Hamiltonian}.
%	Moreover, an explicit computation involving equation \eqref{Eqn: boundary terms} shows that for all $\omega_1=\bigg[\begin{matrix}E_1\\B_1\end{matrix}\bigg]\in\mathrm{dom}(H^*)$ and $\eta=\bigg[\begin{matrix}E_2\\B_2\end{matrix}\bigg]\in\mathrm{dom}(H_\infty)$ it holds
%	\begin{align}\label{Eqn: useful identity}
%	(H^*\omega,\eta)-(\omega,H^*\eta)=
%	-_{-\frac{1}{2}}\left\langle
%	\frac{i}{\sqrt{2}}
%	\bigg[\begin{matrix}
%	\iota_-\ast_\Sigma[\mathrm{t}B_1]\\
%	\iota_-\ast_\Sigma[\mathrm{t}E_1]
%	\end{matrix}\bigg]\,,\,
%	\frac{i}{\sqrt{2}}
%	\bigg[\begin{matrix}
%	\iota_+\ast_\Sigma(\mathrm{t}_-+\mathrm{t}_+)B_2\\
%	\iota_+\ast_\Sigma(\mathrm{t}_-+\mathrm{t}_+)E_2
%	\end{matrix}\bigg]
%	\right\rangle_{\frac{1}{2}}=	
%	-(\gamma_0\omega,\gamma_1\eta)_{\Omega_{\mathrm{Tr}}^1(Z)}\,.
%	\end{align}
%	The latter equality is exploited to prove equation \eqref{Eqn: boundary triple relation}.
%	Indeed for $\omega_j=\bigg[\begin{matrix}E_j\\B_j\end{matrix}\bigg]\in\mathrm{dom}(H^*)$, $j\in\{1,2\}$, we decompose $\omega_j=\omega_{j,\eta}+\omega_{j,\infty}$ according to decomposition \eqref{Eqn: decomposition of domain of curl-Hamiltonian} -- \textit{cf.} remark \ref{Rmk: trace space for curl-Hamiltonian}.
%	Then, exploiting the self-adjointness of $H_\infty$ we have
%	\begin{align*}
%	(H^*\omega_1,\omega_2)-(\omega_1,H^*\omega_2)&=
%	(H^*\omega_{1,\eta},\omega_{2,\infty})+
%	(H^*\omega_{1,\infty},\omega_{2,\eta})-
%	(\omega_{1,\eta},H^*\omega_{2,\infty})-
%	(\omega_{1,\infty},H^*\omega_{2,\eta})\\&=
%	(\gamma_1\omega_1,\gamma_0\omega_2)_{\Omega_{\mathrm{Tr}}^1(Z)}-
%	(\gamma_0\omega_1,\gamma_1\omega_2)_{\Omega_{\mathrm{Tr}}^1(Z)}\,,
%	\end{align*}
%	where in the last equality we used equation \eqref{Eqn: useful identity}.
%\end{proof}
%\begin{remark}
%	According to theorem \ref{Thm: boundary triple}, the results of proposition \ref{Proposition: boundary triple for curl-Hamiltonian} -- see also remark \ref{Rmk: spaces with no jumps} -- implies that the self-adjoint extension $H^*|_{\ker\gamma_0}$ coincides with the operator $H_\infty$.
%	\\\nicorrection{Not really true because of the normal jump!}
%\end{remark}