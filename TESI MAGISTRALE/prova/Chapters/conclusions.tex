\chapter*{Conclusions}
\label{conclusions}
\addcontentsline{toc}{chapter}{Conclusions}
\thispagestyle{plain}
In this thesis we described the local and global well-posedness of wave-like equations propagating on a curved globally hyperbolic Lorentzian manifold, using the method of fundamental solutions.\\

\noindent In particular we outlined the construction of two fundamental solutions $G_+$ and $G_-$, the \emph{retarded} and the \emph{advanced} one, that propagate the source of the equation respectively in the causal future and in the causal past, in accordance with the causality principle.\\
In Chapter \ref{chapter2}, we addressed the problem of solving the wave operator on Minkowski spacetime via the theory of Fourier transform. This relies on the existence of a translation group, a feature which is indeed enjoyed by Minkowski spacetime. Therefore, they are not well-suited for a generic curved background, such as Schwarzschild or cosmological spacetimes, which lack in general of such symmetries.\\
We showed explicit formulas for the wave operator on Minkowsi spacetime with dimensions $2\leq n\leq 4$ with Fourier transform. Then to deduce general properties of solutions, we introduced the \emph{Riesz distributions}, a family of distributions $R_\pm(\alpha)$ with which we equivalently solved the problem, but that are more suitable for further generalizations.
We addressed the Cauchy problem on flat space, realizing that it is well posed if the initial data are set on a particular class of hypersurfaces: Cauchy hypersurfaces (see Definition \ref{defn:Cauchyhyper}).\\
\noindent In Chapter \ref{chapter3} we focus on the wave propagation problem on a suitably Lorentzian manifold. In particular, we extended the Riesz distributions locally on generic Lorentzian manifolds and combined them in order to obtain local fundamental solutions. Gluing together local solutions, we subsequently found that if the manifold respects the condition of \emph{global hyperbolicity}, which is equivalent to the request of the existence of a Cauchy hypersurface, we can construct a global and smooth solution to the Cauchy problem, maintaining the local support properties.\\
Globally hyperbolic Lorentzian manifolds turned out to form a good class for the solution theory of wave-like operators. On them we have unique advanced and retarded fundamental solutions and the global Cauchy problem is well-posed.\\[1.5cm]




% As we mentioned, the Fourier transform approach is unmanageable when dealing with wave-like differential equations on curved backgrounds, hence, to reach our target, we followed another path based on local extensions of Riesz distributions from tangent space to the manifold.\\
%In fact, initially we pulled back Riesz distribution from the tangent space of a fixed point $x\in\mM$ on a geodesically starshaped open subset centered on $x$. Then we made the following formal ansatz: to find local retarded and advanced fundamental solutions for a wave-like operator one must look for infinite formal combinations of selected Riesz distributions, with coefficients that are functions $V_x^k$ to be determined for any $k\in\mathbb{N}$. Implementing this condition formally, one finds recursive relations for the coefficients. It turned out that such relations are differential equations that can be always solved. The formal series had to be transformed into a true fundamental solution. We got rid of the difference between the asymptotic series and the fundamental solution and we found a true local fundamental solution with methods of functional analysis.\\
%At this point we were able to solve the local Cauchy problem: we deduced that if we fix smooth initial data on a Cauchy surface of a manifold, the local solution is unique, smooth and its support is included in the causal past and future of the union of the supports of the initial data. In other words, we proved that a wave-like signal cannot locally travel faster than light.\\
\thispagestyle{plain}
%\noindent Gluing together local solutions, we found that if the manifold respects the condition of \emph{global hyperbolicity} (Definition \ref{defn:globalhyp}), which is equivalent to the request of the existence of a Cauchy hypersurface, we can construct a global and smooth solution to the Cauchy problem, maintaining the local support properties that we mentioned earlier. Globally hyperbolic Lorentzian manifolds turned out to form a good class for the solution theory of wave-like operators. On them we have unique advanced and retarded fundamental solutions and Green's operators, i.e. operators that are the inverse of a differential operator $P$ (see Definition \ref{defn:green}). The Cauchy problem is well-posed.\\
%In conclusion, we made clear how retarded and advanced fundamental solutions are strictly related to Green's operator. In fact they can be seen as two aspects of mainly the same concept.\\[1.5cm]
\thispagestyle{empty}
\noindent The possible extensions and follow-ups of this work are many. Firstly, the results can be extended to differential operators that act on sections of vector bundles, in other words, to vector valued fields, such as the vector potential in electromagnetism. This opens the door of quantum fields theory on curved backgrounds, whose aim is to provide a partial unification of General Relativity with Quantum Physics where the gravitational field is left classical while the other fields are quantized\footnote{see [\citealp{hack}].}. In particular one can deal with other operators such as Dirac operator $D$, whose square is a wave-like operator, as well as with free electrodynamics, both starting from the vector potential or from the Faraday tensor..
Other extensions may go in the direction of addressing the global problem in non-globally hyperbolic manifolds, since most basic models in
General Relativity turn out to be globally hyperbolic, but there are exceptions such as
anti-deSitter spacetime.

\thispagestyle{plain}

