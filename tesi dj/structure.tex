%----------------------------------------------------------------------------------------
%	VARIOUS REQUIRED PACKAGES AND CONFIGURATIONS
%----------------------------------------------------------------------------------------
%\usepackage{tocbibind}
\usepackage{braket}
\usepackage{epigraph}
%\usepackage{fancyhdr}
\usepackage[colorlinks=true,linkcolor=blue]{hyperref}
\usepackage[utf8]{inputenc}
\usepackage[T1]{fontenc} % Support for more character glyphs
\usepackage[round, numbers]{natbib}\citeindextrue % Round brackets around citations, change to square for square brackets
\usepackage{graphicx} % Required to include images
\usepackage{color} % Required for custom colors
\usepackage{amsmath,amssymb, amsfonts} % Math packages
\usepackage{listings} % Required for including snippets of code
\usepackage{booktabs} % Required for better horizontal rules in tables
\usepackage{xspace} % Provides the ability to use an intelligent space which is used in \institution and \department
\usepackage{tabularx}
\usepackage[printonlyused,withpage]{acronym} % Include a list of acronyms
\usepackage{rotating} % Allows tables and figures to be rotated
%\usepackage{hyperref} % Required for links and changing link options
\usepackage{microtype} % Slightly tweak font spacing for aesthetics
\usepackage{amsthm}
\usepackage{mathrsfs}
\usepackage{enumitem}
\usepackage{tikz-cd}
\usepackage{wrapfig}
%\usetikzlibrary{decorations.text}
%\usepackage[x11names]{xcolor}
%\usepackage{subcaption}
\usepackage{dsfont}
\usepackage{lipsum}
\usepackage{longtable}
\usepackage{siunitx}
\usepackage{subfiles}
\graphicspath{{images/}{../images/}}


\usepackage{lmodern} % Use the Palatino font by default
\renewcommand\rmdefault{ptm}



\usepackage[top=3cm,bottom=6cm,headsep=10pt,inner=4cm,outer=3cm, a4paper]{geometry}

\renewcommand*\contentsname{Index}

\usepackage{afterpage}

\newcommand\blankpage{%
	\null
	\thispagestyle{empty}%
	%\addtocounter{page}{-1}%
	\newpage}



\usepackage{fancyhdr}
\pagestyle{fancy}
\fancyhead[RO,LE]{\leftmark}
\fancyhead[RE,LO]{\rightmark}

\renewcommand{\chaptermark}[1]{\markboth{{\thechapter.\ #1}}{}}
\renewcommand{\sectionmark}[1]{ \markright{#1}{} }

%\hypersetup{colorlinks, breaklinks, linkcolor=black,citecolor=black,filecolor=black,urlcolor=black} % Set up hyperlinks including colors for references, urls and citations

%\definecolor{c64}{rgb}{.063,0,.612} % Example color definition, the color can be used with the \color{name} command

\makeatletter
\renewcommand{\fnum@figure}{\textsc{\figurename~\thefigure}} % Make the "Figure 1.1" text in small caps
\makeatother




%----------------------------------------------------------------------------------------
%	MATH THEOREM DEFINITIONS
%----------------------------------------------------------------------------------------





\newtheoremstyle{classicthm}% Nome
{12pt}% Spazio che precede l’enunciato
{12pt}% Spazio che segue l’enunciato
{\slshape}% Stile del font dell’enunciato
{}% Rientro (se vuoto, non c’è rientro,
% \parindent = rientro dei capoversi)
{\bfseries}% Stile del font dell’intestazione
{.}% Punteggiatura che segue l’intestazione
{.5em}% Spazio che segue l’intestazione:
% " " = normale spazio inter-parola;
% \newline = a capo
{}% Specifica l’intestazione dell’enunciato
% (normalmente viene lasciata vuota)







\swapnumbers
\theoremstyle{classicthm}

\newtheorem{theoremd}{Theorem}[section]
\newenvironment{theorem}{\begin{theoremd}}{ \end{theoremd}}


\newtheorem{theoremd*}{Theorem}
\newenvironment{theorem*}{\begin{theoremd*}}{ \end{theoremd*}}
\newtheorem{*theoremd}[theoremd]{$^*$Theorem}
\newenvironment{*theorem}{\begin{*theoremd}}{ \end{*theoremd}}
\newtheorem{cord}[theoremd]{Corollary}

\newenvironment{cor}{\begin{cord}}{ \end{cord}}
\newtheorem{*cor}[theoremd]{$^*$Corollary}
\newtheorem{**cor}[theoremd]{$^{**}$Corollary}
\newtheorem{lemd}[theoremd]{Lemma}
\newenvironment{lem}{\begin{lemd}}{ \end{lemd}}

\newtheorem{*lemd}[theoremd]{$^*$Lemma}
\newenvironment{*lem}{\begin{*lemd}}{ \end{*lemd}}



\newtheorem{propd}[theoremd]{Proposition}

\newenvironment{prop}{\begin{propd}}{ \end{propd}}
\newtheorem{*propd}[theoremd]{$^*$Proposition}
\newenvironment{*prop}{\begin{*propd}}{ \end{*propd}}
\newtheorem{**prop}[theoremd]{$^{**}$Proposition}
\newtheorem{axiom}[theoremd]{Axiom}
\newtheorem{axiom*}{Axiom}


\newtheorem{definitiond}[theoremd]{Definition}
\newenvironment{definition}{\begin{definitiond}}{ \end{definitiond}}
\newtheorem{*definitiond}[theoremd]{$^*$Definition}
\newenvironment{*definition}{\begin{*definitiond}}{ \end{*definitiond}}
\newtheorem{property}[theoremd]{Property}

\newtheorem{notation*}{Notation}
\newtheorem{notation}[theoremd]{Convention}
%\newtheorem{Postulato}{Postulato}






\theoremstyle{definition}
\newtheorem{ossd}[theoremd]{Observation}
\newenvironment{oss}{\begin{ossd}}{ \end{ossd}}
\newtheorem{exercised}[theoremd]{Exercise}
\newenvironment{exercise}{\begin{exercised}}{\end{exercised}}
\newenvironment{esercise*}{\begin{exercised}}{\end{exercised}}
%\newenvironment{comment}{\begin{oss}}{\endproof \end{oss}}
\newtheorem{*oss}[theoremd]{$^*$Observation}
\newenvironment{*comment}{\begin{*oss}}{ \end{*oss}}
\newenvironment{comment*}{\begin{oss}}{ \end{oss}}
\newtheorem{exampled}[theoremd]{Example}
\newenvironment{example}{\begin{exampled}}{
	\end{exampled}}
	\newtheorem{*exampled}[theoremd]{$^*$Example}
	\newenvironment{*example}{\begin{*exampled}}{	\end{*exampled}}
	\newtheorem{remd}[theoremd]{Remark} % Defines the remark environment
	\newenvironment{rem}{\begin{remd}}{	\end{remd}}
	\newtheorem{noted}{Note}[theoremd] % Defines the note environment
	\newenvironment{note}{\begin{noted}}{	\end{noted}}
	
	%\hypersetup{backref,pdfpagemode=FullScreen, colorlinks=true}
	
	
	
	
	
	
	
	
	
	%%------------------------------------------------------
	
	%		DRAWINGS
	
	%%-------------------------------------
	
	\usepackage{tikz}
	\usetikzlibrary{calc}
	\usepackage{pgfplots}
	\usepackage{tikz-3dplot}
	\usetikzlibrary{decorations.markings}
	\usetikzlibrary{shapes,arrows}
	\newcommand{\midarrowright}{\tikz \draw[-triangle 90] (0,0) -- +(.1,0);}
	\newcommand{\midarrowup}{\tikz \draw[-triangle 90] (0,0) -- +(0,.1);}
	
	\usetikzlibrary{shadings, calc, decorations.markings}
	\tikzset{->-/.style={decoration={
				markings,
				mark=at position #1 with {\arrow{>}}},postaction={decorate}},
		->-/.default=0.5,
	}
	
	
	
	
	
	
	
	%-----------------------------------------------------
	
	%		MY FUNCTIONS
	
	%%-----------------------------------------
	
	
	\newcommand\ds{\displaystyle}
	\newcommand\ts{\textstyle}
	\newcommand{\mb}{\mathbf}
	%\renewcommand{\thenotation}{}
	%\renewcommand{\theequation}{\thesection.\arabic{equation}}
	\def\Caption #1{\caption{\footnotesize #1}}
	%\renewcommand\Caption{#1}{\Caption{\small{#1}}}
	%\def\Caption #1{\Caption{\small{{#1}}}}
	
	\def \Bfemph #1{\textbf{\emph{#1}}}
	
	
	%\def\Proof.{{\medbreak\noindent{\it Dimostrazione}\enspace}}
	\def\Proof{{\medbreak\noindent{\textbf{Proof.} }}}
	\def\Proofsketch{{\medbreak\noindent{\textbf{Sketch of proof.} }}}
	\def\endproof{~\hfill $\blacksquare$\par\bigskip}
	%\def\endproof{\hfill$\square$\par\medskip}
	
	\def\Svolgimento{{\medbreak\noindent{\textit{Execution.} }}}
	\def\Suggerimento{{\medbreak\noindent{\textit{Hint:} }}}
	
	
	\def\mR{{\mathbb R}}
	\def\mC{{\mathbb C}}
	\newcommand{\parz}[2]{ \frac{\partial #1}{\partial #2}}
	\newcommand{\deri}[2]{\displaystyle \frac{\dd #1}{\dd #2}}
	\renewcommand{\Re}{\text{Re }}
	\renewcommand{\Im}{\text{Im }}
	%\renewcommand{\theta}{\vartheta}
	\newcommand{\Int}{\text{Int }}
	\newcommand{\Ext}{\text{Ext }}
	\newcommand{\supp}{\text{supp}}
	\newcommand{\mD}{\mathcal{D}}
	\newcommand{\dd}{\mathrm d}
	\newcommand{\norm}[1]{\displaystyle \left \| #1 \right \|}
	\renewcommand{\div}{\operatorname{div}}
	\newcommand{\rot}{\operatorname{rot}}
	\newcommand{\grad}{\operatorname{grad}}
	\newcommand{\id}{\mathds{1}}
	\newcommand{\mM}{\mathrm{M}}
	\newcommand{\mT}{\mathrm{T}}
	%\renewcommand{\to}{\longrightarrow}
	\newcommand{\scalar}[2]{\left\langle #1, #2 \right\rangle}
	\newcommand{\mf}[1]{\mathbf{#1}}
	\newcommand{\nspace}{\!\!\!}
	
	
	\def\Xint#1{\mathchoice 
		{\XXint\displaystyle\textstyle{#1}}% 
		{\XXint\textstyle\scriptstyle{#1}}% 
		{\XXint\scriptstyle\scriptscriptstyle{#1}}% 
		{\XXint\scriptscriptstyle\scriptscriptstyle{#1}}% 
		\!\int} 
	\def\XXint#1#2#3{{\setbox0=\hbox{$#1{#2#3}{\int}$} 
			\vcenter{\hbox{$#2#3$}}\kern-.5\wd0}} 
	\def\Mint{\Xint -}
	
	
	\renewcommand{\hat}[1]{\widehat{#1}}
	\renewcommand{\theta}{\vartheta}
	\renewcommand{\epsilon}{\varepsilon}
	%\renewcommand{\phi}{\varphi}
	\newcommand{\res}{\mathop{\mathrm{Res }}}
	
	\newcommand{\colonna}[2]{\begin{pmatrix}
			#1 \\ #2
		\end{pmatrix}}
		\newcommand{\riga}[2]{\begin{pmatrix}
				#1 & #2
			\end{pmatrix}}
			%%%%%%%%%%%%%%%%%%%%%%%%%%%%%%%%%%%%%%%%%%%%%%%%%%%%%%%%%%%%%%%%%%%%%%%%%%%%%%%%%%%%%%%%%%%
			
			%----------------------------------------------------------------------------------------
			%	CODE SNIPPET CONFIGURATION
			%----------------------------------------------------------------------------------------
			
			\lstset{
				basicstyle=\ttfamily\small,
				basewidth=0.55em,
				showstringspaces=false,
				numbers=left,
				numberstyle=\tiny,
				numbersep=2.5pt,
				keywordstyle=\bfseries\ttfamily,
				breaklines=true
			}
			% Examples of list environments for different programming languages, you will likely need to specify your own
			\lstnewenvironment{pseudoc}{\lstset{frame=lines,language=C,mathescape=true}}{}
			\lstnewenvironment{logs}{\lstset{frame=lines,basicstyle=\footnotesize\ttfamily,numbers=none}}{}
			\lstnewenvironment{cc}{\lstset{frame=lines,language=C}}{}
			\lstnewenvironment{c64}{\lstset{backgroundcolor=\color{c64},basewidth=0.65em,basicstyle=\commodoreface\color{c64light},numbers=none,framerule=10pt,rulecolor=\color{c64light},frame=tb,framexbottommargin=30pt}}{}
			\lstnewenvironment{html}{\lstset{frame=lines,language=html,numbers=none}}{}
			\lstnewenvironment{pseudo}{\lstset{frame=lines,mathescape=true,morekeywords={learn_string_domain, save_model}}}{}
			\lstnewenvironment{pseudoctiny}{\lstset{language=C,mathescape=true,basicstyle=\tiny\sffamily}}{}
			\lstnewenvironment{cctiny}{\lstset{language=C,basicstyle=\tiny\sffamily}}{}
			\lstnewenvironment{pseudotiny}{\lstset{mathescape=true,basicstyle=\tiny\sffamily}}{}